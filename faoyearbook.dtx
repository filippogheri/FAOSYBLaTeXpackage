% \iffalse
% $Id: faoyearbook.dtx,v 1.11 2011-08-18 22:28:15 boris Exp $
%
% Copyright 2011-2013, Food and Agriculture Organization of the  United Nations
% \fi 
%
% \CheckSum{2721}
%
%% \CharacterTable
%%  {Upper-case    \A\B\C\D\E\F\G\H\I\J\K\L\M\N\O\P\Q\R\S\T\U\V\W\X\Y\Z
%%   Lower-case    \a\b\c\d\e\f\g\h\i\j\k\l\m\n\o\p\q\r\s\t\u\v\w\x\y\z
%%   Digits        \0\1\2\3\4\5\6\7\8\9
%%   Exclamation   \!     Double quote  \"     Hash (number) \#
%%   Dollar        \$     Percent       \%     Ampersand     \&
%%   Acute accent  \'     Left paren    \(     Right paren   \)
%%   Asterisk      \*     Plus          \+     Comma         \,
%%   Minus         \-     Point         \.     Solidus       \/
%%   Colon         \:     Semicolon     \;     Less than     \<
%%   Equals        \=     Greater than  \>     Question mark \?
%%   Commercial at \@     Left bracket  \[     Backslash     \\
%%   Right bracket \]     Circumflex    \^     Underscore    \_
%%   Grave accent  \`     Left brace    \{     Vertical bar  \|
%%   Right brace   \}     Tilde         \~} 
%
%\iffalse
% Taken from xkeyval.dtx and amsdtx.dtx
%\fi
%\makeatletter
%\def\DescribeOption#1{\leavevmode\@bsphack
%              \marginpar{\raggedleft\PrintDescribeOption{#1}}%
%              \SpecialOptionIndex{#1}\@esphack\ignorespaces}
%\def\DescribeKey#1{\leavevmode\@bsphack
%              \marginpar{\raggedleft\PrintDescribeKey{#1}}%
%              \SpecialKeyIndex{#1}\@esphack\ignorespaces}
%\def\PrintDescribeOption#1{\strut\emph{option}\\\MacroFont #1\ }
%\let\HDorg@PrintDescribeOption\PrintDescribeOption
%\def\PrintDescribeKey#1{\strut\emph{key}\\\MacroFont #1\ }
%\def\SpecialOptionIndex#1{\@bsphack
%    \index{#1\actualchar{\protect\ttfamily#1}
%           (option)\encapchar usage}%
%    \index{options:\levelchar#1\actualchar{\protect\ttfamily#1}\encapchar
%           usage}\@esphack}
%\let\HDorg@SpecialOptionIndex\SpecialOptionIndex
%\def\SpecialKeyIndex#1{\@bsphack
%    \index{#1\actualchar{\protect\ttfamily#1}
%           (key)\encapchar usage}%
%    \index{keys:\levelchar#1\actualchar{\protect\ttfamily#1}\encapchar
%           usage}\@esphack}
%\def\DescribeOptions#1{\leavevmode\@bsphack
%  \marginpar{\raggedleft\strut\emph{options}%
%  \@for\@tempa:=#1\do{%
%    \\\strut\MacroFont\@tempa\SpecialOptionIndex\@tempa
%  }}\@esphack\ignorespaces}
% \def\cls#1{\texttt{#1}}
% \def\cs#1{\textbackslash\texttt{#1}}
% \let\cn\cs
%\makeatother
% \newcommand{\progname}[1]{\textsf{#1}}
%
% \MakeShortVerb{|}
% \GetFileInfo{faoyearbook.dtx}
% \title{\LaTeX{} Style for FAO Yearbook
%   \thanks{\copyright 2011, Food and Agriculture Organization of the
%   United Nations}}
% \author{Boris Veytsman\thanks{%
% \href{mailto:borisv@lk.net}{\texttt{borisv@lk.net}},
% \href{mailto:boris@varphi.com}{\texttt{boris@varphi.com}}}} 
% \date{\filedate, \fileversion}
% \maketitle
% \begin{abstract}
%   This package provides class for typesetting FAO Yearbook
% \end{abstract}
% \tableofcontents
% \listoffigures
% \clearpage
%
%\section{Introduction}
%\label{sec:intro}
%  
% FAO UN publishes yearbooks of statistical data.  This class is
% intended to typeset them.  It was commissioned by FAO, and the
% copyright rests with UN.
%
%
%\section{User Guide}
%\label{sec:user_guide}
%
%\subsection{Installation}
%\label{sec:ug_install}
% 
% The class relies on the following software and packages installed on
% your system:
% \begin{enumerate}
% \item Fonts and \LaTeX{} font support packages
% \progname{arev}~\cite{Hartke:ArevSans} and
% \progname{paratype}~\cite{Farar:ParaType}. 
% \item \progname{graphics} bundle~\cite{Carlisle05:Graphics}.
% \item \progname{xcolor} package~\cite{Kern07:Xcolor}.
% \item \progname{fancyhdr} package~\cite{Oostrum04:Fancyhdr}.
% \item \progname{TikZ} and \progname{PGF} packages~\cite{Tantau:TikZ}.
% \item \progname{geometry} package~\cite{Umeki08:Geometry}.
% \item \progname{paralel} and \progname{pdfcolparallel}
% packages~\cite{Eckermann:Parallel, Oberdiek:Pdfcolparallel}.
% \item \progname{float} and \progname{caption}
% packages~\cite{Lingnau:Float, Sommerfeldt07:Caption}.
% \item \progname{booktabs} and \progname{longtable}
% packages~\cite{Fear05:Booktabs, Carlisle04:Longtable}.
% \item \progname{lscape} package~\cite{Carlisle:Lscape}.
% \item \progname{xkeyval} package~\cite{Adriaens:Xkeyval}.
% \item \progname{array} package~\cite{Mittelbach:Array}.
% \item \progname{multicol} package~\cite{Mittelbach:Multicol}.
% \item \progname{hhline} package~\cite{Carlisle:Hhline}.
% \end{enumerate}
% 
% The installation of the class follows the usual
% practice~\cite{TeXFAQ} for \LaTeX{} packages:
% \begin{enumerate}
% \item Run \progname{latex} on |faoyearbook.ins|.  This will produce the
% \LaTeX{} class |faoyearbook.cls|.
% \item Put the file |faoyearbook.cls| to the place where \LaTeX{} can
%   find it (see \cite{TeXFAQ} or the documentation for your \TeX{}
%   system).\label{item:install}
% \item Update the database of file names.  Again, see \cite{TeXFAQ}
% or the documentation for your \TeX{} system for the system-specific
% details.\label{item:update}
% \item The file |faoyearbook.pdf| provides the documentation for the
% package (this is the file you are probably reading now).
% \end{enumerate}
% As an alternative to items~\ref{item:install} and~\ref{item:update}
% you can just put the file |faoyearbook.cls| in the working directory
% where your |.tex| file is.
%
%\subsubsection{Packages troubleshooting}

%\paragraph{Paratype} Installing the PT sans font from package \progname{paratype} can prove difficult. 
%On windows, this should not make any major problem with MikTex. On Linux on the other side, 
%it be be advantageous to use the debian package: \url{http://www.bilbo.dynip.com/debian/dists/woody/updates/binary-all/paratype_20110509-1_all.deb}.
%In case a message such as: 
%\begin{verbatim}!pdfTeX error: pdflatex (file ectt3583): Font ectt3583 at 600 not found \end{verbatim} arises, 
%one should (re) install the package \progname{ec}, at best from the \progname{cm-super} bundle available in Debian/Ubuntu repos.
%\paragraph{Geometry} Another issue that can happen is a message like:
%\begin{verbatim}! Missing number, treated as zero.
%<to be read again> 
%                   \Gm@cnth 
%\end{verbatim}
%This is due to an outdated version of \progname{geometry}. Version 5.6 will work.
%\subsection{Invocation}
%\label{sec:ug_invocation}
%
% To use the class, put in the preamble of your document
% \begin{flushleft}
% |\documentclass[|\meta{options}|]{faoyearbook}|
% \end{flushleft}
%
% \DescribeOption{print}
% \DescribeOption{web}
% If the option |web| (default) is chosen, the pages of the book have
% the dimensions corresponding to A4 paper.  However, if the option
% |print| is chosen, then the pages are printed on a wider area, and
% crop marks are added for the trimming.
%
% \DescribeOption{issuu}
% If the option |issuu| is chosen, the internal links are transformed
% to external in the form suitable for \url{http://www.issuu.com}.
% Note that this option probably does not make much sense unless |web|
% option is also chosen.  However, it is still possible to select both
% |print| and |issuu| option if someone needs it for an obscure
% purpose.  
%
% \DescribeMacro{altMargins}
% \DescribeMacro{altMarginsNarrow}
% Some regional books are printed on A4 paper with slightly different
% margins.  These options select alternative margins---either normal
% or narrow.  Note that these options automatically select |print|
% setup, and should not be combined with the |print| option.
%
%
% \DescribeOption{Draft}
% \DescribeOption{draft}
% The option |Draft| (note the capitalization!) leads to the the large
% word `DRAFT' printed across the pages.  The standard \LaTeX{} option
% |draft| leads to the same result, but it also makes other changes,
% most notably, in the behavior of the |\includegraphics| command and
% warnings.  
%
% \DescribeMacro{\ifprint}
% It is possible to query the current mode using the macro
% \cmd{\ifprint}, for example
% \begin{verbatim}
% \ifprint
%   Stuff for print version
% \else
%   Stuff for web version
% \fi
% \end{verbatim}
% Any branch of this conditional may be empty, so web-only stuff can
% be coded as
% \begin{verbatim}
% \ifprint\else Web-only stuff\fi
% \end{verbatim}
% 
%
% \DescribeMacro{\includegraphics}
% There is a special facilty for \cmd{\includegraphics} command to
% choose a file depending on the current mode of the package.  Namely,
% if there is a file |image_print.pdf| visible by \LaTeX, then the commands
% |\includegraphics{image}| or  |\includegraphics{image.pdf}| selects
% the file |image_print.pdf|.  In the case this file is not found, the
% file |image.pdf| is selected instead.  Similarly in the web mode the
% file |image_web.pdf| will be selected first, and only if it does not
% exist, |image.pdf| is selected.  This rule works also for commands
% \cmd{\includeLargeGraphics} and \cmd{\includeExtraLargeGraphics}
% described below.
%
% Note that at this time there is no
% similar facility for the |\input| command.  
%
%\subsection{Setting Parameters}
%\label{sec:ug_faoset}
%
% \DescribeMacro{\faoset}
% Some parameters in the class can be set with the command
% |\faoset|\marg{key=value}, for example
% \begin{verbatim}
% \faoset{bgcolor=blue}
% \end{verbatim}
% 
%
%  Most of the parameters are explained below.
%
%\subsection{Fonts}
%\label{sec:ug_fonts}
%
%\DescribeMacro{\narrowfamily}
%\DescribeMacro{\textnarrow}
%\DescribeMacro{\captionfamily}
%\DescribeMacro{\textcaption}
% The class uses PT Sans fonts~\cite{Farar:ParaType} for body text and
% Arev fonts~\cite{Hartke:ArevSans} for math.  It defines two
% additional families: Narrow and Caption, corresponding to the PT
% Sans Narrow and PT Sans Caption font.  They can be selected by the
% declarations \cmd{\narrowfamily} and \cmd{\captionfamily} or by the
% commands \cmd{\textnarrow}\marg{text} and
% \cmd{\textcaption}\marg{text} following the usual \LaTeX{}
% conventions.  Note that since PT Sans does not provide math
% alphabet, this choice does not change the mathematical text.
%
% PT Sans Narrow may be useful for typesetting tables, for example,
% \begin{verbatim}
% \begin{tablepages}
%  \scriptsize\narrowfamily
%  \rowcolors{4}{@bgcolor!30}{@bgcolor!20}
%  \input{./Tables/P1.DEM_1.tex}
% \end{tablepages}
% \end{verbatim}
%  
%
%\subsection{Setting Color}
%\label{sec:ug_color}
%
% \DescribeMacro{\setbgcolor} 
% Each part of the Yearbook is typeset
% with it own background color.  It is used a theme in the ``special
% sections'' (see below), indicator pages, etc.  The color is selected
% with the macro |\setbgcolor|\oarg{color}, where \oarg{color} can be
% in any form acceptable by \progname{xcolor}
% package~\cite{Kern07:Xcolor}, for example
% \begin{verbatim}
% \setbgcolor{blue}
% \setbgcolor{-green!25!blue!60}
% \setbgcolor{rgb,9:red,4;green,2;yellow,1}
% \end{verbatim}
% The command must be issued \emph{before} the corresponding |\part|
% command. 
%
% \DescribeKey{bgcolor}
% An alternative way is to use the key-value interface with the
% command |\faoset|:
% \begin{verbatim}
% \faoset{bgcolor=rgb,9:red,4;green,2;yellow,1}
% \end{verbatim}
%
% \DescribeKey{tableheadcolor}
% The key |tableheadcolor| sets the color for the headers of tables defined
% by |H| or |P| key (see Section~\ref{sec:ug_floats}).  Normally it is
% 30\% of the current |bgcolor| color, but it can be set to any required
% value. 
%
% \DescribeMacro{\selectcolor}
% The command |\setbgcolor| by itself does not change the background
% color: it is changed by the next |\part| macro.  However, if by eny
% reason you need to change the color \emph{before} the next |\part|,
% you can do it using the command |\selectcolor|, which is similar to
% the |\selectfont| command of the standard \LaTeX.  In most cases the
% user probably should not employ this command directly.
%
%
%
%\subsection{Setting spacing}
%\label{sec:ug_spacing}
%
% \DescribeKey{tocpartskip}
% If necessary user can make the table of contents stretch using the
% parameter |tocpartskip| (by default, zero), for example:
% \begin{verbatim}
% \faoset{tocpartskip=12pt plus 12pt}
% \end{verbatim}
% Not that this command must be issued \emph{before}
% \cmd{\tableofcontents}.  This parameter gives the extra space before
% the Part entries in the table of contents.
%
%\subsection{Sectioning}
%\label{sec:ug_sections}
%
% \DescribeMacro{\part} 
% \DescribeMacro{\section} 
% \DescribeMacro{\subsection} 
%  The main division of the text are |\part|s.
%  The command \cmd{\part}\marg{title} is used for numbered parts, while the
%  command \cmd{\part*}\marg{title} is used for unnumbered parts.  The
%  next division are |\section|s and 
% |\subsection|s.  They are never numbered.  The style does not use
% |\chapter|s. 
%
% \DescribeMacro{\EndPartIntro}
% The sections immediately following new parts are special:  they are
% typeset in one column on specially colored pages.  The command
% |\EndPartIntro| switches to the ``normal'' sections.  
%
%
% \DescribeMacro{\appendix}
% Some parts of the text belong to \emph{appendices.}  The usual
% command |\appendix| should be issued before these parts.  The parts
% typed after this command have special formatting---in particular,
% they do not produce colored bands in the table of contents.
%
%
%\subsection{(Non)Floats}
%\label{sec:ug_floats}
%
% Illustrative materials in \LaTeX{} are usually called floats because
% they ``float'' from page to page, and \TeX{} processor finds the
% best place for them.  We still use the same name, but our floats
% \emph{do not} float: instead they are pinned to the place where they
% are defined (so ``nonfloats'' might be a better name for them).
% Normally they occur in the graphics part of the indicator pages (see
% Section~\ref{sec:ug_indicator}), but the class does not produce an
% error if they are in any other place of the book.  There are three
% types of floats defined by the class:
% \DescribeEnv{table}
% \DescribeEnv{chart}
% \DescribeEnv{map}
% \DescribeEnv{minitab}
% \begin{description}
% \item[table] tabular material,
% \item[chart] plots and other charts,
% \item[map] mapped data.
% \item[minitab] mini tables.
% \end{description}
% \DescribeMacro{caption}
% Each of these kinds of material is typeset using the corresponding
% environment: |table|, |chart|, |minitab| or |map|.  Note that the caption for
% each of these environments \emph{must} precede the graphical
% material, for example:
% \begin{verbatim}
% \begin{chart}
%   \centering
%   \caption{Hunger Data}
%   \label{chart:hunger}
%   \includegraphics{hunger.pdf}\\
% \end{chart}
% \end{verbatim}
%
% \DescribeMacro{\footnotebar}
% The source for the graphics sometimes is put after the graphics
% itself.  In this case the command \cmd{\footnotebar} produces a
% bar separating the source from the graphics, for example
% \begin{verbatim}
% \begin{chart}
%   \centering
%   \caption{Hunger Data}
%   \label{chart:hunger}
%   \includegraphics{hunger.pdf}
%   \footnotebar
%    Source: FAO, Statistics Division, WEB: \url{http://www.fao.org} 
% \end{chart}
% \end{verbatim}
%
%
% \DescribeMacro{\listoftables}
% \DescribeMacro{\listofcharts}
% \DescribeMacro{\listofmaps}
% The standard \LaTeX{} has the command |\listoftables| to produce the
% list of tables in the document.  Our class retains this command and
% produces two additional commands |\listofcharts| and |\listofmaps|
% with the obvious meaning.
%
%
% 
% \DescribeKey{floatwidth}
% Normally each (non)float besides long tables is set in the box with
% the width equal to one column of text and the height equal to the
% natural height of the float (including the caption).  The width of
% the float can be changed with the key |floatwidth|:
% \begin{verbatim}
% \faoset{floatwidth=0.5\columnwidth}
% \end{verbatim}
% \DescribeKey{fixedfloatheight}
% For gridded floats it is useful to make the floats to have the same
% height to align in the adjacent columns.  To achieve this effect one
% can set the Boolean key |fixedfloatheight|, normally |false|, to |true|:
% \begin{verbatim}
% \faoset{fixedfloatheight=true}
% \end{verbatim}
% \DescribeKey{floatheight}
%  By default the height of the float is chosen to be 0.45 of
%  the height of the page (two floats per column).  One can override
%  this decision by setting |floatheight| to another value, for
%  example, to get three plots per column, one sets
% \begin{verbatim}
% \faoset{floatheight=0.3\textheight}
% \end{verbatim}
% \DescribeKey{fixedcaptionheight}
% \DescribeKey{captionheight}
%  For gridded floats it is useful to typeset the caption in a box of
%  fixed height, so the graphics is aligned.  When the Boolean key
%  |fixedcaptionheight|, normally |false|, is set to |true|, the
%  captions are typeset to fixed height, which is equal to
%  |captionheight| (normally equal to |2\baselinskip|).  For example,
%  if all captions are one line long, one can use
% \begin{verbatim}
% \faoset{fixedcaptionheight=true, captionheight=2\baselineskip}
% \end{verbatim}
%  
%
%
%\subsection{Tables}
%\label{sec:ug_tables}
%
%
%
% \DescribeEnv{tablepages}
% Long tables at the end of sections are typeset in the special style:
% no headers of footers, slightly colored pages, small margins.  This
% style is selected by the environment |tablepages|:
% \begin{verbatim}
% \begin{tablepages}
%   TABLES
% \end{tablepages}
% \end{verbatim}
% 
%
%
% \DescribeKey{tablebg}
% Normally the color selected by the |tablepage| style is the same as the
% main color of the section.  However, it can be changed by the
% command \cmd{faoset} using the key |tablebg|, for example
% \begin{verbatim}
% \faoset{tablebg=blue!5}
% \pagestyle{tablepage}
% \end{verbatim}
% 
%
%
%  
% The rules for typesetting tables can be found in \progname{booktabs}
% and \progname{longtable} packages.  Here we just remind the rules:
% \begin{enumerate}
% \item A table starts with |\toprule| and ends with |\bottomrule|.
% The header is separated with |\midrule|.
% \item For |longtable|s the commands |\endfirsthead| and |\endhead|
% denotes the header on the first and subsequent pages.
% \item The command |\endfoot| dentes the footer repeated on every page.
% \item Caption can be set on the first header.  The command
% |\caption[]|\marg{Caption} is used on the subsequent heads.
% \end{enumerate}
% By default the rows of a table are colored using the background
% color of the current part. 
%
% \DescribeKey{d}
% \DescribeKey{H}
% \DescribeKey{P}
% \DescribeKey{C}
% To typeset numericall items one should use |d| column identifier
% with the format |d|\marg{a.b}, where $a$ is the number of decimal
% in the integer part of the number, and $b$ is the number of decimal
% digitst in the fractional part.  For example, a number $12.345$
% corresponds to |d{2.3}|.  The column headers are usually \emph{not}
% numerical, so one need to use \cmd{\multicolumn} entries to typeset
% them.  The class defines several such entries:
% \begin{description}
% \item[H] produces a centered entry.
% \item[P] produces an entry of a given length, for example,
% |P{1.5cm}|
%
% \item[C] produces an entry of the length corresponding to 
% the given number of numerical columns.  For example, |C{2}|
% corresponds to a header of two numerical columns.  Each column is
% assumed to be of the size enough to store $-99.999$.  
% \end{description}
% 
% \DescribeMacro{\hhline}
% Unfortunately, the command \cmd{\cmidrule} from the
% \progname{booktabs} package is not working well with the colored
% headings: the columns that do not the rule display a white band
% instead.  Therefore for the rules that do not span the table width
% \cmd{\hhline}\marg{specificaiton} command from the \progname{hhline}
% package should be used.  The \marg{specification} argument of this
% command has many variants, but for our purposes we need only one
% variant: the command |-| produces a horizontal line spanning one
% column.  The color of this line is determined by the command
% \cmd{\arrayrulecolor}\marg{color}, issued in the last |>|\marg{argument}
% command before the |-| specification.  Therefore the command
% |>{\arrayrulecolor{@tableheadcolor}}-| produces a line of the color
% |@tableheadcolor|, which is seen as the absence of line.  The
% command |>{\arrayrulecolor{black}}---| produces a black line
% spanning three columns. Thus if we have a four-column table and want
% a rule spanning columns 2--3, the following command should be issued:
% \begin{verbatim}
%   \hhline{>{\arrayrulecolor{@tableheadcolor}}-% Column 1, no rule
%     >{\arrayrulecolor{black}}--%  Columns 2 and 3, black rule
%     >{\arrayrulecolor{@tableheadcolor}}-}% Column 4, no rule
% \end{verbatim}
% The usual |*| specification may be used for repeating patterns, for
% example, |*{5}{-}| is equivalent to |-----|.  
%
% The vertical bar \verb+|+ specification in the \cmd{\hhline}
% argument means an interruption of the line. The interruption is by
% defalut a black interval, to make it the same color as the header
% background, use \verb+>{\arrayrulecolor{@tableheadcolor}}|+. 
%
%
%\subsection{Large Maps}
%\label{sec:ug_large_graphics}
%
%
%
% \DescribeMacro{\includeLargeGraphics}
% \DescribeMacro{\includeExtraLargeGraphics}
% There is a special case of very large maps that can take either
% three or even four columns of width. They can be typeset in a
% spread, when the left part takes a recto page while the right part
% takes a verso page of an indicator (see
% Section~\ref{sec:ug_indicator}).  The special commands
% |\includeLargeGraphics|\marg{file} and
% |\includeExtraLargeGraphics|\marg{file} are used to typeset three
% column and four column maps correspondingly.  Since they take care
% to align the graphics vertically and assume the existence of a
% caption, they do not work outside a (non)float.
%
% A three-column map can be included inside |indicator| (see
% Section~\ref{sec:ug_indicator}) in the following way:
% \begin{verbatim}
% \begin{inicator}
%   \begin{indicatorText}
%     Text part
%   \end{indicatorText}
%   \begin{indicatorGraphics}
%     \begin{map}
%       \caption{A Caption}
%       \label{map:ALargeMap}
%       \includeLargeGraphics{large_map.pdf}
%     \end{map}
%   \end{indicatorGraphics}
% \end{inicator}
% \end{verbatim}
% For a four-column map we need |indicatorC|:
% \begin{verbatim}
% \begin{inicatorC}
%   \begin{indicatorGraphics}
%     \begin{map}
%       \caption{A Caption}
%       \label{map:AVeryLargeMap}
%       \includeExtraLargeGraphics{very_large_map.pdf}
%     \end{map}
%   \end{indicatorGraphics}
% \end{inicatorC}
% \end{verbatim}
%
% \DescribeKey{leftfraction}
% Normally the commands \cmd{\includeLargeGraphics} and
% \cmd{\includeExtraLargeGraphics} leave $0.36$ or $0.5$ of the
% image on the left and put the rest of the image on the right.
% However, you can change this using the key |leftfraction|, for
% example, the command
% \begin{verbatim}
% \includeLargeGraphics[leftfraction=0.4]{large_map.pdf}
% \end{verbatim}
% splits the image as $4:6$.
%
% \DescribeMacro{\largeGraphicsNotes}
% All the material typed before \cmd{\includeLargeGraphics} is typeset
% before the graphics material.  All the material typed after
% \cmd{\includeLargeGraphics} is typeset in the recto page after the
% right part of the graphics.  However, sometimes one need to typeset
% some material on the verso page after the left part of the map.  The
% command \cmd{\largeGraphicsNotes}\marg{material} should be issued
% before the graphics.  The \marg{material} is saved and typeset after
% the graphics, for example
% \begin{verbatim}
% \begin{map}
%   \caption{Caption}
%   \largeGraphicsNotes{% Material to be typeset on the left
%     \begin{itemize}
%     \item One
%     \item Two
%     \end{itemize}}
%   \includeLargeGraphics{map}
%   Material to be typeset on the right
% \end{map}
% \end{verbatim}
% 
% \DescribeEnv{graphicKey}
% Sometimes a large map requires a small rectangular key.  The
% environment |graphicKey| typesets its contents at the right lower
% corner of the map.  If it contains more than an
% \cmd{\includegraphics} command, you need to tell \TeX{} the width of
% the corresponding box.  For this you can use the optional parameter,
% for example
% \begin{verbatim}
% \begin{graphicKey}
%   \includegraphics{key}
% \end{graphicKey}
% or
% \begin{graphicKey}[4cm]
%   \includegraphics{key}\\
%   This is the key
% \end{graphicKey}
% \end{verbatim}
% Note that the key should be \emph{outside} the |map| environment,
% otherwise \TeX{} cannot position it properly.
%
% \DescribeKey{lefparttoffset}
% \DescribeKey{rightpartoffset}
% Sometimes the parts of the large graphics must be slighly moved to
% the left or to the right.  The keys |leftpartoffset| and
% |rightpartoffset| allow this manipulation, for example
% \begin{verbatim}
% \includeLargeGraphics[leftpartoffset=-1cm, 
%        rightpartoffset=1cm]{large_map.pdf}
% \end{verbatim}
% The default values for \cmd{\includeLargeGraphics} are
% \begin{verbatim}
% leftpartoffset=-20pt, rightpartoffset=0pt
% \end{verbatim}
% 
%
%
%\subsection{Indicator Pages}
%\label{sec:ug_indicator}
%
% The main illustrative device for the book is called
% \emph{indicator}.  Indicator pages consist of text and illistrative
% material (tables, graphs, maps).  The text and the illustrations are
% visually separated and flow independently.
%
%
%
% \begin{figure}
%   \centering
%   \begin{picture}(250,200)
%     \put(0,50){\framebox(50,150){Text}}
%     \put(50,50){\framebox(50,150){\color{blue}Graphics}}
%     \put(30,25){Verso page}
%     \put(150,50){\framebox(50,150){Text}}
%     \put(200,50){\framebox(50,150){\color{blue}Graphics}}
%     \put(180,25){Recto page}
%   \end{picture}
%   \caption{Indicator, Variant~A}
%   \label{fig:indicatorA}
% \end{figure}
% 
%
%
% \begin{figure}
%   \centering
%   \begin{picture}(250,200)
%     \put(0,50){\framebox(50,150){Text}}
%     \put(50,50){\framebox(50,150){\color{blue}Graphics}}
%     \put(30,25){Verso page}
%     \put(150,50){\framebox(50,150){\color{blue}Graphics}}
%     \put(200,50){\framebox(50,150){\color{blue}Graphics}}
%     \put(180,25){Recto page}
%   \end{picture}
%   \caption{Indicator, Variant~B$_1$}
%   \label{fig:indicatorB1}
% \end{figure}
%
% \begin{figure}
%   \centering
%   \begin{picture}(250,200)
%     \put(0,50){\framebox(50,150){Text}}
%     \put(50,50){\framebox(50,150){\color{blue}Graphics}}
%     \put(30,25){Verso page}
%     \put(150,50){\framebox(100,150){\color{blue}Graphics}}
%     \put(180,25){Recto page}
%   \end{picture}
%   \caption{Indicator, Variant~B$_2$}
%   \label{fig:indicatorB2}
% \end{figure}
%
%
%
% \begin{figure}
%   \centering
%   \begin{picture}(250,200)
%     \put(0,50){\framebox(50,150){\color{blue}Graphics}}
%     \put(50,50){\framebox(50,150){\color{blue}Graphics}}
%     \put(30,25){Verso page}
%     \put(150,50){\framebox(50,150){\color{blue}Graphics}}
%     \put(200,50){\framebox(50,150){\color{blue}Graphics}}
%     \put(180,25){Recto page}
%   \end{picture}
%   \caption{Indicator, Variant~C$_1$}
%   \label{fig:indicatorC1}
% \end{figure}
%
%
% \begin{figure}
%   \centering
%   \begin{picture}(250,200)
%     \put(0,50){\framebox(50,150){\color{blue}Graphics}}
%     \put(50,50){\framebox(50,150){\color{blue}Graphics}}
%     \put(30,25){Verso page}
%     \put(150,50){\framebox(100,150){\color{blue}Graphics}}
%     \put(180,25){Recto page}
%   \end{picture}
%   \caption{Indicator, Variant~C$_2$}
%   \label{fig:indicatorC2}
% \end{figure}
%
%
% \begin{figure}
%   \centering
%   \begin{picture}(250,200)
%     \put(0,50){\framebox(100,150){\color{blue}Graphics}}
%     \put(30,25){Verso page}
%     \put(150,50){\framebox(50,150){\color{blue}Graphics}}
%     \put(200,50){\framebox(50,150){\color{blue}Graphics}}
%     \put(180,25){Recto page}
%   \end{picture}
%   \caption{Indicator, Variant~C$_3$}
%   \label{fig:indicatorC3}
% \end{figure}
%
% \begin{figure}
%   \centering
%   \begin{picture}(250,200)
%     \put(0,50){\framebox(100,150){\color{blue}Graphics}}
%     \put(30,25){Verso page}
%     \put(150,50){\framebox(100,150){\color{blue}Graphics}}
%     \put(180,25){Recto page}
%   \end{picture}
%   \caption{Indicator, Variant~C$_4$}
%   \label{fig:indicatorC4}
% \end{figure}
%
% \begin{figure}
%   \centering
%   \begin{picture}(250,200)
%     \put(0,125){\framebox(50,75){Text}}
%     \put(50,125){\framebox(50,75){Text}}
%     \put(0,50){\framebox(50,75){\color{blue}Graphics}}
%     \put(50,50){\framebox(50,75){\color{blue}Graphics}}
%     \put(30,25){Verso page}
%     \put(150,50){\framebox(50,150){\color{blue}Graphics}}
%     \put(200,50){\framebox(50,150){\color{blue}Graphics}}
%     \put(180,25){Recto page}
%   \end{picture}
%   \caption{Indicator, Variant~D$_1$}
%   \label{fig:indicatorD1}
% \end{figure}
%
% \begin{figure}
%   \centering
%   \begin{picture}(250,200)
%     \put(0,125){\framebox(50,75){Text}}
%     \put(50,125){\framebox(50,75){Text}}
%     \put(0,50){\framebox(100,75){\color{blue}Graphics}}
%     \put(30,25){Verso page}
%     \put(150,50){\framebox(50,150){\color{blue}Graphics}}
%     \put(200,50){\framebox(50,150){\color{blue}Graphics}}
%     \put(180,25){Recto page}
%   \end{picture}
%   \caption{Indicator, Variant~D$_2$}
%   \label{fig:indicatorD2}
% \end{figure}
%
% \begin{figure}
%   \centering
%   \begin{picture}(250,200)
%     \put(0,125){\framebox(50,75){Text}}
%     \put(50,125){\framebox(50,75){Text}}
%     \put(0,50){\framebox(50,75){\color{blue}Graphics}}
%     \put(50,50){\framebox(50,75){\color{blue}Graphics}}
%     \put(30,25){Verso page}
%     \put(150,50){\framebox(100,150){\color{blue}Graphics}}
%     \put(180,25){Recto page}
%   \end{picture}
%   \caption{Indicator, Variant~D$_3$}
%   \label{fig:indicatorD3}
% \end{figure}
% 
% \begin{figure}
%   \centering
%   \begin{picture}(250,200)
%     \put(0,125){\framebox(50,75){Text}}
%     \put(50,125){\framebox(50,75){Text}}
%     \put(0,50){\framebox(100,75){\color{blue}Graphics}}
%     \put(30,25){Verso page}
%     \put(150,50){\framebox(100,150){\color{blue}Graphics}}
%     \put(180,25){Recto page}
%   \end{picture}
%   \caption{Indicator, Variant~D$_4$}
%   \label{fig:indicatorD4}
% \end{figure}
%
% An indicator always starts at a verso (even-numbered) page.  Most of
% the indicators run for a spread (even-odd pair of pages).  There are
% several variants of indicator pages as shown on
% Figures~\ref{fig:indicatorA}--\ref{fig:indicatorC4}.  In variant~A
% (Figure~\ref{fig:indicatorA}) there are two columns: text and
% graphics, running on each page of the indicator.  This design
% continues as needed until both text and illustrations are exhausted.
% In variant~B (Figures~\ref{fig:indicatorB1}
% and~\ref{fig:indicatorB2}) text is present only on the first page of
% the indicator.  The next pagee of the indicator have only
% illustrative material: either in two column mode
% (Figure~\ref{fig:indicatorB1}) or in one column mode
% (Figure~\ref{fig:indicatorB2}).  In variant~C
% (Figures~\ref{fig:indicatorC1}--\ref{fig:indicatorC4}) there is no
% text part at all: graphics takes all pages, either in one column or
% in two column mode. In variant~D the verso page is divided
% horizontally into text part and graphics part.  The graphics part on
% the page can have one large graphics or two smaller, as well as the
% the graphics part on the recto page can be one or two column wide
% (see Figures~\ref{fig:indicatorD1}--\ref{fig:indicatorD4}).  
%
%
% \DescribeEnv{indicatorA}
% \DescribeEnv{indicatorB}
% \DescribeEnv{indicatorC}
% \DescribeEnv{indicatorD}
% The implementation of all these variants is done with the
% corresponding \LaTeX{} environments: |indicatorA|, |indicatorB| and
% |indicatorC|.   The switch from two column and one column layout in
% the \emph{graphics} part in variants~B and~C is done with the usual
% \LaTeX{} commands 
% |\onecolumn| and |\twocolumn|.  Note that these commands in the
% \emph{text} part may lead to strange results.  Similarly, you should
% not use them in any part of variant~A indicator.  
%
%
% \DescribeEnv{indicatorText}
% \DescribeEnv{indicatorGraphics}
% The text and graphics parts are typeset using the environments
% |indicatorText| and |indicatorGraphics|.  The first environment
% should always precede the second one; of course, there is not text
% part in variant~C.
%
% For example, the layout in variant~A is achieved in the following
% way: 
% \begin{verbatim}
% \begin{indicatorA}
%   \begin{indicatorText}
%     Several pages of text material
%   \end{indicatorText}
%   \begin{indicatorGraphics}
%     Several pages of graphics material
%   \end{indicatorGraphics}
% \end{indicatorA}
% \end{verbatim}
% 
% while the layout in variant~B$_2$ is achieved in the following way:
% \begin{verbatim}
% \begin{indicatorB}
%   \begin{indicatorText}
%     A page of text material
%   \end{indicatorText}
%   \begin{indicatorGraphics}
%     First page of illustrations
%     \onecolumn
%     Second page of illustrations
%   \end{indicatorGraphics}
% \end{indicatorB}
% \end{verbatim}
%
% and the layout in variant~C$_4$ in the follwing way:
% \begin{verbatim}
% \begin{indicatorC}
%   \begin{indicatorGraphics}
%     \onecolumn
%     Several pages of illustrations
%   \end{indicatorGraphics}
% \end{indicatorC}
% \end{verbatim}
%
%
%  Up to now we discussed only the indicators that occupy a
%  spread---two consecqutive pages.  What happens if we have material
%  for more than one spread?  The answer is simple in variants~A
%  and~C:  the patterns there can continue for an unlimited number of
%  pages.  The variant~B is special: we have a mixed text-graphics
%  page, then a graphics-only page.  Should the third page be graphics
%  only or a mixed one?  In our interface it can be both.  Namely, if
%  the graphics material continues for several pages, they will be
%  graphics-only pages.  However, the pair
%  |indicatorText|-|indicatorGraphics| can be repeated, and in this
%  case we start another spread like the ones on
%  figures~\ref{fig:indicatorB1} and~\ref{fig:indicatorB2}.
%
%  In the following example we have one mixed page and three graphics
%  page:
% \begin{verbatim}
% \begin{indicatorB}
%   \begin{indicatorText}
%     A page of text material
%   \end{indicatorText}
%   \begin{indicatorGraphics}
%     Three pages of illustrations
%   \end{indicatorGraphics}
% \end{indicatorB}
% \end{verbatim}
%  while in the following example we have a mixed page, a page of
%  graphics material, a mixed page and another graphics page:
% \begin{verbatim}
% \begin{indicatorB}
%   \begin{indicatorText}
%     A page of text material
%   \end{indicatorText}
%   \begin{indicatorGraphics}
%     Three columns of illustrations
%   \end{indicatorGraphics}
%   \begin{indicatorText}
%     A page of text material
%   \end{indicatorText}
%   \begin{indicatorGraphics}
%     Three columns of illustrations
%   \end{indicatorGraphics}
% \end{indicatorB}
% \end{verbatim}
%
%  \DescribeEnv{indicator} \DescribeEnv{indicatorPage} Actually
%  |indicatorB| is so important, that aliases |indicator| and
%  |indicatorPage| can be used for it:
% \begin{verbatim}
% \begin{indicator}
%   \begin{indicatorText}
%     A page of text material
%   \end{indicatorText}
%   \begin{indicatorGraphics}
%     Three columns of illustrations
%   \end{indicatorGraphics}
%   \begin{indicatorText}
%     A page of text material
%   \end{indicatorText}
%   \begin{indicatorGraphics}
%     Three columns of illustrations
%   \end{indicatorGraphics}
% \end{indicator} 
% \end{verbatim}
% 
%
%
%  It should be said that such interface decision involves a fair
%  amount of visual formatting.  The user must decide how to
%  distribute the material between the pages.  A more \LaTeX ish
%  solution would be to let \TeX{} decide where to make the page
%  breaks.  It \emph{is} possible to do, but it is more difficult and
%  it seems that the book design envisions visual formatting anyway.
%  Therefore we, in \progname{beamer}-like manner, leave the
%  distribution of material between the frames to the
%  author.
%
%
%
%\subsection{Special Graphics Layouts}
%\label{sec:ug_special_layouts}
% 
% 
% While the combination of |\onecolumn| and |\twocolumn| can produce
% any layout of plots, there are several predefined layouts that help
% to produce the most common ones.  They are described in this
% section.  
%
% \DescribeEnv{TwoPlusFourPlots} 
% The $2+4$ layout
% (Figure~\ref{fig:TwoPlusFourPlots}) combines two plots on the left
% page of the indicator spread and four plots on the right page:
% \begin{verbatim}
% \begin{indicator}
%   \begin{indicatorText}
%     Text Part
%   \end{indicatorText}
%   \begin{indicatorGraphics}
%     \begin{TwoPlusFourPlots}
%       \begin{chart}
%         First chart
%       \end{chart}
%       \begin{chart}
%         Second chart
%       \end{chart}
%       ....
%       \begin{chart}
%         Last chart
%       \end{chart}
%     \end{TwoPlusFourPlots}
%   \end{indicatorGraphics}
% \end{indicator}
% \end{verbatim}
% 
%
% \begin{figure}
%   \centering
%   \begin{picture}(250,200)
%     \put(0,50){\framebox(50,150){Text}}
%     \put(50,50){\framebox(50,150){}}
%     \put(60,80){\framebox(30,30){\color{blue}Plot}}
%     \put(60,140){\framebox(30,30){\color{blue}Plot}}
%     \put(30,25){Verso page}
%     \put(150,50){\framebox(100,150){}}
%     \put(160,80){\framebox(30,30){\color{blue}Plot}}
%     \put(160,140){\framebox(30,30){\color{blue}Plot}}
%     \put(210,80){\framebox(30,30){\color{blue}Plot}}
%     \put(210,140){\framebox(30,30){\color{blue}Plot}}
%     \put(180,25){Recto page}
%   \end{picture}
%   \caption{$2+4$ Layout}
%   \label{fig:TwoPlusFourPlots}
% \end{figure}
%
% \DescribeEnv{TwoPlusTwoPlots}
% Similarly |TwoPlusTwoPlots| has two one-column plots  on the left
% and two two-column plots on the right
% (Figure~\ref{fig:TwoPlusTwoPlots}).
%
%
% \begin{figure}
%   \centering
%   \begin{picture}(250,200)
%     \put(0,50){\framebox(50,150){Text}}
%     \put(50,50){\framebox(50,150){}}
%     \put(60,80){\framebox(30,30){\color{blue}Plot}}
%     \put(60,140){\framebox(30,30){\color{blue}Plot}}
%     \put(30,25){Verso page}
%     \put(150,50){\framebox(100,150){}}
%     \put(180,80){\framebox(40,30){\color{blue}Plot}}
%     \put(180,140){\framebox(40,30){\color{blue}Plot}}
%     \put(180,25){Recto page}
%   \end{picture}
%   \caption{$2+2$ Layout}
%   \label{fig:TwoPlusTwoPlots}
% \end{figure}
%
% \DescribeEnv{OnePlusTwoPlots}
% A tall plot can be accomodated in |OnePlusTwoPlots| layout
% (Figure~\ref{fig:OnePlusTwoPlots}). 
%
%
% \begin{figure}
%   \centering
%   \begin{picture}(250,200)
%     \put(0,50){\framebox(50,150){Text}}
%     \put(50,50){\framebox(50,150){}}
%     \put(60,80){\framebox(30,90){\color{blue}Plot}}
%     \put(30,25){Verso page}
%     \put(150,50){\framebox(100,150){}}
%     \put(180,80){\framebox(40,30){\color{blue}Plot}}
%     \put(180,140){\framebox(40,30){\color{blue}Plot}}
%     \put(180,25){Recto page}
%   \end{picture}
%   \caption{$1+2$ Layout}
%   \label{fig:OnePlusTwoPlots}
% \end{figure}
%
%
% Last, it is possible to put three plots on a recto page as shown of
% Figures~\ref{fig:TwoPlusThreePlotsA}
% and~\ref{fig:TwoPlusThreePlotsB}.  For this one can use
% |TwoPlusTwoPlots| combined with |multicols| environment as in the
% following example:
% \begin{verbatim}
% \begin{TwoPlusTwoPlots}
%   \begin{chart}
%     First chart on the verso page
%   \end{chart}
%
%   \begin{chart}
%     Second chart on the verso page
%   \end{chart}
%
%   \begin{chart}
%     First chart on the recto page
%   \end{chart}
%
%   \begin{multicols}{2}
%     \begin{chart}
%       Second chart on the recto page
%     \end{chart}
%
%     \begin{chart}
%       Third chart on the recto page
%     \end{chart}
%   \end{multicols}
% \end{TwoPlusTwoPlots}
% \end{verbatim}
% 
%
% \begin{figure}
%   \centering
%   \begin{picture}(250,200)
%     \put(0,50){\framebox(50,150){Text}}
%     \put(50,50){\framebox(50,150){}}
%     \put(60,80){\framebox(30,30){\color{blue}Plot}}
%     \put(60,140){\framebox(30,30){\color{blue}Plot}}
%     \put(30,25){Verso page}
%     \put(150,50){\framebox(100,150){}}
%     \put(160,80){\framebox(30,30){\color{blue}Plot}}
%     \put(210,80){\framebox(30,30){\color{blue}Plot}}
%     \put(180,140){\framebox(40,30){\color{blue}Plot}}
%     \put(180,25){Recto page}
%   \end{picture}
%   \caption{$2+3$ Layout, Variant A}
%   \label{fig:TwoPlusThreePlotsA}
% \end{figure}
%
%
% \begin{figure}
%   \centering
%   \begin{picture}(250,200)
%     \put(0,50){\framebox(50,150){Text}}
%     \put(50,50){\framebox(50,150){}}
%     \put(60,80){\framebox(30,30){\color{blue}Plot}}
%     \put(60,140){\framebox(30,30){\color{blue}Plot}}
%     \put(30,25){Verso page}
%     \put(150,50){\framebox(100,150){}}
%     \put(160,140){\framebox(30,30){\color{blue}Plot}}
%     \put(210,140){\framebox(30,30){\color{blue}Plot}}
%     \put(180,80){\framebox(40,30){\color{blue}Plot}}
%     \put(180,25){Recto page}
%   \end{picture}
%   \caption{$2+3$ Layout, Variant B}
%   \label{fig:TwoPlusThreePlotsB}
% \end{figure}
%
%
%
%\subsection{Publication Descriptions}
%\label{sec:ug_publications}
%
% \DescribeEnv{publication} FAO yearboook describes some FAO
% publications.  These publications should be put inside the
% environment |publication|.  The environment has one mandatory
% argument, which is the title of the publication, and one optional
% argument, which sets the file name of the publication cover. Note
% that the option argument, if present, must precede the mandatory
% one. If this
% argument is absent, no cover is included.
% \DescribeMacro{\pDescription} 
% \DescribeMacro{\pEdition} \DescribeMacro{\pCycle} 
% \DescribeMacro{pWeb}
% Inside the
% environment the macros \cmd{\pDescription}\marg{description},
% \cmd{\pEdition}\marg{year}\marg{edition}, \cmd{\pWeb}\marg{URL}
% and \cmd{\pCycle}\marg{date} are used to typeset the corresponding
% items related to the publication.  For example,
% \begin{verbatim}
% \begin{publication}[./Plots/StateOfFoodAndAgriculture.png]{The State
%     of Food and Agriculture}
%   \pDescription{The State of Food and Agriculture, FAO's major
%     annual flagship publication, aims at bringing to a wider
%     audience balanced science-based assessments of important issues
%     in the field of food and agriculture.  Each edition of the
%     report contains a comprehensive, yet easily accessible, overview
%     of a selected topic of major relevance for rural and
%     agricultural development and for global food security.  This is
%     supplemented by a synthetic overview of the current global
%     agricultural situation.}
%  \pEdition{2010}{Livestock in the balance}
%  \pEdition{2011}{Women in Agriculture Closing the gender gap for
%    development} 
%  \pCycle{May each year}
%  \pWeb{http://www.fao.org/docrep/013/i2050e/i2050e00.htm}
% \end{publication}
% \end{verbatim}
% Note that, as in the example, some fields may be repeated.  
%
% \DescribeKey{publicationparskip}
% \DescribeKey{publicationskip}
% Two spacing parameters can be used for typesetting of publications:
% |publicationskip| is the amount of additional space between the publications,
% while |publicationparskip| is the space between the paragraphs
% inside the publication environment.  The default values correspond
% to the command
% \begin{verbatim}
% \faoset{publicationskip=6pt plus 2pt minus 2pt,
%         publicationparskip=6pt plus 6pt minus 4pt}
% \end{verbatim}
% 
% 
%
%\subsection{Metadata}
%\label{sec:ug_metadata}
%
% \DescribeEnv{MetadataCollection}
% \DescribeEnv{metadata}
% Each chart, map of table in the book has a \emph{source}.  Soruces
% are collected in the environemnt |MetadataCollection|, which consists
% of separate
% |metadata| environments.  Each |metadata| environment  has two obligatory
% arguments---the name of the source and the key.  The key is used to
% identify the metadata in the charts, maps, tables and other
% objects.  The environment may include other commands.
% \begin{description}
% \item[\cmd{\source}\marg{source}]\DescribeMacro{\source} sets the
% source of the data.
% \item[\cmd{\owner}\marg{owner}]\DescribeMacro{\owner} sets the owner
% of the data.
% \end{description}
% Note that there is no ``description'' command because any text which
% is not an argument of the commands above is considered to belong to
% the description of the data.
%
% Example of the usage of these commands:
% \begin{verbatim}
% \begin{MetadataCollection}
% \begin{metadata}{Agricultural population}{P1.DEM.FAO.POP.AGR} 
% 
%    Agricultural population is defined as all persons depending for
%    their livelihood on agriculture, hunting, fishing and forestry.
%    It comprises all persons economically active in agriculture as
%    well as their non-working dependents. It is not necessary that
%    this referred population exclusively come from rural population.
%
%    \source{FILL ME}
%    \owner{FILL ME}
% \end{metadata}
% \end{MetadataCollection}
% \end{verbatim}
%
% \DescribeMacro{\refMetadata}
% The metadata is referenced by the command
% \cmd{\refMetdata}\marg{key}, for example
% \begin{verbatim}
% \refMetadata{P1.DEM.FAO.POP.AGR}
% \end{verbatim}
% This command will be typset as
% \begin{quote}
%   Source: Agricultural population, page NNNN.
% \end{quote}
% This command must \emph{not} occur in the caption of the chart, map
% or table.
%
%
% Note that the package automatically provides backreferencing: all
% charts, maps and tables where the medatada is referenced, are
% mentioned in the corresponding metadata section.
%
% \DescribeKey{metadataInLists}
% The sources of each chart, map or table can be shown in the lists of
% charts, tables, maps or not.  The key |metadataInLists| (by default
% |false|) determines whether they are shown there.  To make them
% visible, put before the lists
% \begin{verbatim}
% \faosetup{metadataInLists=true}
% \end{verbatim}
% 
%
%
%
%\subsection{Concepts and Methods}
%\label{sec:ug_concepts}
% 
% \DescribeEnv{ConceptsAndMethods}
% The environment |ConceptsAndMethods| starts a new section ``Concepts
% and Methods''.
% Concepts and methods are collected in the series of |concept|
% environments. Each environment has one obligatory field: the name of
% the concept, for example:
% \begin{verbatim}
% \begin{ConceptsAndMethods}
%   \begin{concept}{Gross domestic product}
%     Gross domestic product (GDP) is the market value of all officially
%     recognized final goods and services produced within a country in a
%     given period of time. 
%   \end{concept}
%   \begin{concept}{Gross state product}
%     Gross state product (GSP), or gross regional product (GRP), is a
%     measurement of the economic output of a state or province (i.e.,
%     of a subnational entity). It is the sum of all value added by
%     industries within the state and serves as a counterpart to the
%     gross domestic product (GDP). 
%   \end{concept}
% \end{ConceptsAndMethods}
% \end{verbatim}
% 
% 
%
%\subsection{Further Reading}
%\label{sec:ug_further_reading}
% 
% \DescribeEnv{freading}
% The special environment |freading| is used for the ``further
% reading'' sections of the book.  It starts the text from the new
% page and changes some defaults.  
%
%
%\subsection{Subscripts in Text}
%\label{sec:ug_textsubscript}
%
% \DescribeMacro{\textsubscript}
% The standard \LaTeX{} defines \cmd{\textsuperscript}.  The class
% adds a similar \cmd{\textsubscript} command.
%
% 
%
%\StopEventually{%
% \clearpage
% \bibliography{tex}
% \bibliographystyle{unsrt}}
% \clearpage
%\section{Implementation}
%\label{sec:impl}
%
%\subsection{Identification}
%\label{sec:ident}
%
% We start with the declaration who we are.  Most |.dtx| files put
% driver code in a separate driver file |.drv|.  We roll this code into the
% main file, and use the pseudo-guard |<gobble>| for it.
%    \begin{macrocode}
%<class>\NeedsTeXFormat{LaTeX2e}
%<*gobble>
\ProvidesFile{faoyearbook.dtx}
%</gobble>
%<class>\ProvidesClass{faoyearbook}
[2013/03/23 v1.11 Typesetting FAO Yearbook]
%    \end{macrocode}
% And the driver code:
%    \begin{macrocode}
%<*gobble>
\documentclass{ltxdoc}
\usepackage{array,pict2e,graphpap,xcolor,amsmath}
\usepackage{url,amsfonts,hypdoc}
\usepackage{hyperref}
\PageIndex
\CodelineIndex
\RecordChanges
\EnableCrossrefs
\begin{document}
  \DocInput{faoyearbook.dtx}
\end{document}
%</gobble> 
%<*class>
%    \end{macrocode}
%
%\subsection{Options}
%\label{sec:options}
%
% \begin{macro}{\if@factbook}
% \changes{v0.15}{2011/11/15}{Added macro}
%   Fact book requires some special treatment, so we check whether we
%   are actually in a factbook:
%    \begin{macrocode}
\newif\if@factbook
\@ifclassloaded{faofactbook}{\@factbooktrue}{\@factbookfalse}
%    \end{macrocode}
%   
% \end{macro}
%
%
% \changes{v0.24}{2011/11/26}{Added a4paper as the default}
%    \begin{macrocode}
\if@factbook\else
\PassOptionsToPackage{a4paper,twoside}{geometry}
\fi
%    \end{macrocode}
% 
%
% \begin{macro}{\faoyearbook@size@warning}
% The font-changing options are not used in our setup, so we just
% produce a warning:
%    \begin{macrocode}
\long\def\faoyearbook@size@warning#1{%
  \ClassWarning{faoyearbook}{Size-changing option #1 will not be
    honored}}%
\DeclareOption{8pt}{\faoyearbook@size@warning{\CurrentOption}}%
\DeclareOption{9pt}{\faoyearbook@size@warning{\CurrentOption}}%
\DeclareOption{10pt}{\faoyearbook@size@warning{\CurrentOption}}%
\DeclareOption{11pt}{\faoyearbook@size@warning{\CurrentOption}}%
\DeclareOption{12pt}{\faoyearbook@size@warning{\CurrentOption}}%      
%    \end{macrocode}
% \end{macro}
%
%
% \changes{v0.12}{2011/09/28}{Added web and print options}
% \changes{v0.26}{2011/11/22}{Changed behavior or print option for
% factook} 
% \begin{macro}{\ifprint}
% \changes{v0.15}{2011/10/12}{Changed the name of the macro}
%   We have a flag shich shows whether we are in Web or print mode
%    \begin{macrocode}
\newif\ifprint
\printfalse
\DeclareOption{web}{\printfalse}
\DeclareOption{print}{\printtrue
  \if@factbook
  \PassOptionsToPackage{papersize={12cm,21cm},layoutsize={10cm,19cm},
    layouthoffset=1cm,layoutvoffset=1cm,twoside}{geometry}%
  \else
  \PassOptionsToPackage{papersize={230mm,317mm},layout=a4paper,
    layouthoffset=1cm,layoutvoffset=1cm,twoside}{geometry}%
   \fi}
%    \end{macrocode}   
% \end{macro}
%
%
% \begin{macro}{\if@altMargins}
% \changes{v0.42}{2012/11/13}{Added option} 
% Alternative print margins
%    \begin{macrocode}
\newif\if@altMargins
\@altMarginsfalse
\DeclareOption{altMargins}{\printtrue\@altMarginstrue
  \PassOptionsToPackage{papersize={220mm,307mm},layout=a4paper,
    layouthoffset=5mm,layoutvoffset=5mm,twoside}{geometry}
    \setlength{\footskip}{10pt}}
%    \end{macrocode}
% 
% \end{macro}
%
% \begin{macro}{\if@altMarginsNarrow}
% \changes{v0.43}{2012/11/15}{Added option} 
% Yet another pair of alternative print margins
%    \begin{macrocode}
\newif\if@altMarginsNarrow
\@altMarginsNarrowfalse
\DeclareOption{altMarginsNarrow}{\printtrue\@altMarginsNarrowtrue
  \PassOptionsToPackage{papersize={220mm,307mm},layout=a4paper,
    layouthoffset=5mm,layoutvoffset=5mm,twoside}{geometry}
    \setlength{\footskip}{10pt}}
%    \end{macrocode}
% 
% \end{macro}
%
% \changes{v0.20}{2011/11/06}{Added draft option} 
% \begin{macro}{\ifDraft}
% \changes{v0.20}{2011/11/06}{Added macro} 
%   If we are in `Draft' or `draft mode', we print a word `draft'
%   across the page:
%    \begin{macrocode}
\newif\ifDraft
\Draftfalse
\DeclareOption{Draft}{\Drafttrue}
\DeclareOption{draft}{\Drafttrue}
%    \end{macrocode}
%   
% \end{macro}
%
%
% \begin{macro}{\if@issuumode}
% \changes{v0.41}{2012/03/05}{Added macro}
%   Whether we need issuu-style links
%    \begin{macrocode}
\newif\if@issuumode
\@issuumodefalse
\DeclareOption{issuu}{\@issuumodetrue}
%    \end{macrocode}
%   
% \end{macro}
%
%
% All other options are just sent to the main class:
%    \begin{macrocode}
\DeclareOption*{\PassOptionsToClass{\CurrentOption}{report}}
\ProcessOptions\relax
%    \end{macrocode}
% 
%
%\subsection{Loading Class and Packages}
%\label{sec:loading}
%
% \changes{v0.2}{2011/08/24}{Added xkeyval}
% \changes{v0.4}{2011/08/28}{Added hyperref}
% \changes{v0.4}{2011/08/30}{Deleted pdfborder}
% \changes{v0.5}{2011/09/08}{Deleted dependency on transparent
% package}
% \changes{v0.6}{2011/09/09}{Added siunitx package}
% \changes{v0.9}{2011/09/14}{Added multicol package}
% \changes{v0.10}{2011/09/19}{Added dcolumn package}
% \changes{v0.12}{2011/09/28}{Added cmyk color model to xcolor}
% \changes{v0.14}{2011/10/08}{Added hhline package}
% \changes{v0.16}{2011/10/26}{Added inputenc and fontenc}
% \changes{v0.16}{2011/10/28}{Added afterpage}
% \changes{v0.28}{2011/11/24}{Added natbib and pdfpage}
% \changes{v0.38}{2012/02/05}{Added bookmark}
% We start with the base class and some packages
%    \begin{macrocode}
\if@factbook
\LoadClass[10pt,twoside,twocolumn,a4paper]{report}
\else
\LoadClass[10pt,twoside,twocolumn]{report}
\fi
\RequirePackage{graphicx,xkeyval}
\RequirePackage[table,cmyk]{xcolor}
\RequirePackage{tikz,geometry,dcolumn}
\RequirePackage{fancyhdr,pdfcolparcolumns,pdfcolparallel}
\RequirePackage{float,caption,lscape,longtable,siunitx,booktabs}
\RequirePackage{multicol,atbegshi,picture,hhline,afterpage}
\RequirePackage[T1]{fontenc}
\RequirePackage[utf8x]{inputenc}
\RequirePackage{pdfpages}
\RequirePackage[authoryear]{natbib}
\RequirePackage[breaklinks]{hyperref}
\RequirePackage{bookmark}
\if@issuumode
\RequirePackage{issuulinks}
\fi
%    \end{macrocode}
% Options for the hyperef package are set as follows:
%    \begin{macrocode}
\ifprint
\hypersetup{breaklinks,colorlinks=false,pdfborder=0 0 0,    
  pdfauthor={FAO},
  pdfsubject={Statistical Yearbook of the Food And Agricultural Organization for the United Nations},
  pdftitle={Statistical Yearbook of the Food And Agricultural Organization for the United Nations},
  pdfkeywords={FAO, Food Security, Undernourishment, Sustainable agriculture},
  pdfpagelayout=TwoColumnLeft,
  pdfnewwindow=true
}
\else
\hypersetup{breaklinks,colorlinks=false,pdfborder=0 0 0,    
  pdfauthor={FAO},
  pdfsubject={Statistical Yearbook of the Food And Agricultural Organization for the United Nations},
  pdftitle={Statistical Yearbook of the Food And Agricultural Organization for the United Nations},
  pdfkeywords={FAO, Food Security, Undernourishment, Sustainable agriculture},
  pdfpagelayout=TwoColumnRight,
  pdfnewwindow=true
}
\fi
%    \end{macrocode}
%
% 
%
%\subsection{Color}
%\label{sec:colors}
%
% \changes{v0.35}{2011/12/13}{Added color treatment}
% We need to tell the printer that we are using CMYK color model.  The
% following is taken from the |pdfx| package (the package itself is
% not too easy to make work).
%    \begin{macrocode}
\def\@pctchar{\expandafter\@gobble\string\%}
\def\@bchar{\expandafter\@gobble\string\\}
\immediate\pdfobj stream attr{/N 4}  file{FOGRA39L.icc}
\edef\OBJ@CVR{\the\pdflastobj}
\pdfcatalog{/OutputIntents [ <<
  /Type/OutputIntent
  /S/GTS_PDFX
  /OutputCondition (FOGRA39)
  /OutputConditionIdentifier (FOGRA39 \@bchar(ISO Coated v2
   300\@pctchar\space \@bchar(ECI\@bchar)\@bchar))
  /DestOutputProfile \OBJ@CVR\space 0 R
  /RegistryName(http://www.color.org)
 >> ]}
%    \end{macrocode}
%
%\subsection{Key-Value Interface}
% \label{sec:keyval}
%
% \begin{macro}{\faoset}
% \changes{v0.2}{2011/08/24}{Added macro}
%   We define the family |fao| for our keys:
%    \begin{macrocode}
\def\faoset#1{\setkeys{fao}{#1}}
%    \end{macrocode}
%   
% \end{macro}
%
%\subsection{Fonts}
%\label{sec:fonts}
%
% \changes{v0.15}{2011/10/12}{Switched to PTSans}
% We use arev for mathematics:
%    \begin{macrocode}
\RequirePackage{arevmath}
%    \end{macrocode}
% 
% For body text we use PT~Sans:
%    \begin{macrocode}
\def\PTSans@scale{0.95}
\def\PTSansNarrow@scale{0.95}
\def\PTSansCaption@scale{0.95}
\renewcommand{\sfdefault}{PTSans-TLF}
\renewcommand{\familydefault}{\sfdefault}
\renewcommand{\bfdefault}{b}
%    \end{macrocode}
% \begin{macro}{\narrowfamily}
% \changes{v0.15}{2011/10/12}{Introduced macro}
%   We declare a new family, \cmd{\captionfamily}:
%    \begin{macrocode}
\DeclareRobustCommand\narrowfamily{\fontfamily{PTSansNarrow-TLF}\selectfont}
%    \end{macrocode}
% \end{macro}
% \begin{macro}{\textnarrow}
% \changes{v0.15}{2011/10/12}{Introduced macro}
%   And the matching \cmd{\textnarrow} command:
%    \begin{macrocode}
\DeclareTextFontCommand{\textnarrow}{\narrowfamily}
%    \end{macrocode}   
% \end{macro}
% \begin{macro}{\captionfamily}
% \changes{v0.15}{2011/10/12}{Introduced macro}
%   Same with  \cmd{\captionfamily}:
%    \begin{macrocode}
\DeclareRobustCommand\captionfamily{\fontfamily{PTSansCaption-TLF}\selectfont}
%    \end{macrocode}
% \end{macro}
% \begin{macro}{\textcaption}
% \changes{v0.15}{2011/10/12}{Introduced macro}
%   And the matching \cmd{\textcaption} command:
%    \begin{macrocode}
\DeclareTextFontCommand{\textcaption}{\captionfamily}
%    \end{macrocode}   
% \end{macro}
%
% \begin{macro}{\normalsize}
%   The basic size is 9.6pt:
%    \begin{macrocode}
\renewcommand\normalsize{%
   \@setfontsize\normalsize{9.6pt}{\@xiipt}%
   \abovedisplayskip 10\p@ \@plus2\p@ \@minus5\p@
   \abovedisplayshortskip \z@ \@plus3\p@
   \belowdisplayshortskip 6\p@ \@plus3\p@ \@minus3\p@
   \belowdisplayskip \abovedisplayskip
   \let\@listi\@listI}
\normalsize
%    \end{macrocode}   
% \end{macro}
% \begin{macro}{\small}
% \changes{v0.15}{2011/11/14}{Introduced macro}   
% This is the small size:
%    \begin{macrocode}
\renewcommand\small{%
   \@setfontsize\small\@ixpt{10}%
   \abovedisplayskip 8.5\p@ \@plus3\p@ \@minus4\p@
   \abovedisplayshortskip \z@ \@plus2\p@
   \belowdisplayshortskip 4\p@ \@plus2\p@ \@minus2\p@
   \def\@listi{\leftmargin\leftmargini
               \topsep 4\p@ \@plus2\p@ \@minus2\p@
               \parsep 2\p@ \@plus\p@ \@minus\p@
               \itemsep \parsep}%
   \belowdisplayskip \abovedisplayskip}
%    \end{macrocode}
% \end{macro}
%
% \changes{v0.22}{2011/11/14}{Made URLs roman}
% We use |rm| style of URL:
%    \begin{macrocode}
\urlstyle{sf}
%    \end{macrocode}
% 
%
%\subsection{Margins and Paragraphing}
%\label{sec:pars}
%
% \changes{v0.4}{2011/08/28}{Changed margins}
% \changes{v0.5}{2011/09/08}{Changed bottom margins}
% \changes{v0.11}{2011/09/22}{Changed top margins}
% \changes{v0.22}{2011/11/15}{Added special formatting for factbook
% margins}
% \changes{v0.42}{2012/11/13}{Added special formatting for alt
% margins}
% \changes{v0.42}{2012/11/13}{Added special formatting for alt margins
% narrow}
%  We use a4paper.  Note that we want to move this definition to the
%  beginning of the document to catch the options:
%    \begin{macrocode}
\if@factbook
\AtBeginDocument{%
  \geometry{layoutsize={10cm,19cm},
  left=1cm,right=1cm,bottom=1cm,top=1cm,twoside}%
  \setlength\columnsep{10pt}%
\savegeometry{standard}\@twosidetrue}
\else\if@altMargins
\AtBeginDocument{\geometry{layout=a4paper,
  left=2cm,right=1.5cm,bottom=2cm,top=2cm,twoside}%
  \setlength\columnsep{50pt}%
\savegeometry{standard}\@twosidetrue}
\else\if@altMarginsNarrow
\AtBeginDocument{\geometry{layout=a4paper,
  left=1.5cm,right=1cm,bottom=2cm,top=2cm,twoside}%
  \setlength\columnsep{50pt}%
\savegeometry{standard}\@twosidetrue}
\else
\AtBeginDocument{\geometry{layout=a4paper,
  left=2cm,right=2cm,bottom=2cm,top=2cm,twoside}%
  \setlength\columnsep{50pt}%
\savegeometry{standard}\@twosidetrue}
\fi\fi\fi
%    \end{macrocode}
%  
% \begin{macro}{\parindent}
% \begin{macro}{\parskip}
% \changes{v0.11}{2011/11/04}{Decreased}
%   We use not indented paragraphs with paragraph borders given by
%   skips
%    \begin{macrocode}
\setlength\parindent\z@
\setlength\parskip{6\p@ plus 6\p@ minus 4\p@}
%    \end{macrocode}
%   
% \end{macro}
% \end{macro}
%
%
% 
%
%\subsection{Cropmarks}
%\label{sec:cropmarks}
%
% There are several packages that provide crop marks.  Unfortunately
% they do not work for us because they put crop marks at the
% background.  Since we have colored pages, we want crop marks to be
% on the foreground.
%
% In this section we re-implement cropmarks of the \progname{geometry}
% package, putting the marks on the foreground.
%
% We postpone the code to the beginning of the document to get the
% proper value of the switch
%    \begin{macrocode}
\AtBeginDocument{\ifprint
  \AtBeginShipout{%
    \AtBeginShipoutUpperLeftForeground{%
      \color{black}%
      \@tempdima=\Gm@layouthoffset
      \@tempdimb=\Gm@layoutvoffset
      \put(\@tempdima,-\@tempdimb+6\p@){\line(0,1){50}}%
      \put(\@tempdima-6\p@,-\@tempdimb){\line(-1,0){50}}%
      \advance\@tempdima by \Gm@layoutwidth
      \put(\@tempdima,-\@tempdimb+6\p@){\line(0,1){50}}%
      \put(\@tempdima+6\p@,-\@tempdimb){\line(1,0){50}}%
      \advance\@tempdimb by \Gm@layoutheight
      \put(\@tempdima,-\@tempdimb-6\p@){\line(0,-1){50}}%
      \put(\@tempdima+6\p@,-\@tempdimb){\line(1,0){50}}%
      \advance\@tempdima by -\Gm@layoutwidth
      \put(\@tempdima-6\p@,-\@tempdimb){\line(-1,0){50}}%
      \put(\@tempdima,-\@tempdimb-6\p@){\line(0,-1){50}}%
    }}\fi}
%    \end{macrocode}
% 
% In draft mode we put the word `DRAFT' across the page:
%    \begin{macrocode}
\AtBeginDocument{\ifDraft
  \AtBeginShipout{%
    \AtBeginShipoutUpperLeftForeground{%
      \color{black}%
      \@tempdima=\Gm@layouthoffset
      \@tempdimb=\Gm@layoutvoffset
      \advance\@tempdima by 0.2\Gm@layoutwidth
      \advance\@tempdimb by 0.7\Gm@layoutheight
      \put(\@tempdima,-\@tempdimb){%
        \rotatebox{45}{%
          \fontsize{5cm}{5cm}\selectfont 
          \tikz\node[opacity=0.25,inner sep=\z@]{DRAFT};}}}}\fi}
%    \end{macrocode}
% 
% 
% 
%\subsection{Page Styles}
%\label{sec:page_styles}
% 
% \begin{macro}{\fao@partblobtop}
% \changes{v0.36}{2011/12/19}{Introduced macro}   
% \begin{macro}{\fao@partblobbottom}
% \changes{v0.36}{2011/12/19}{Introduced macro}   
%   Some pages have ``part blobs'': colored blobs on the specific
%   positions of the page.  These macros set the top and the bottom of
%   the blob corresponding to the part set in the second parameter:
%    \begin{macrocode}
\def\fao@partblobtop#1#2{\expandafter\gdef\csname fao@blobstart#1\endcsname{#2}}
\def\fao@partblobbottom#1#2{\expandafter\gdef\csname fao@blobend#1\endcsname{#2}}
%    \end{macrocode}
% \end{macro}
% \end{macro}
%
%
% \begin{macro}{\fao@blobposition}
% \changes{v0.36}{2011/12/27}{Introduced macro}   
% \begin{macro}{\fao@blobheight}
% \changes{v0.36}{2011/12/27}{Introduced macro}   
%  These lengths keep the current position and length of the colored blob:
%    \begin{macrocode}
\newlength{\fao@blobposition}
\newlength{\fao@blobheight}
%    \end{macrocode}
% \end{macro}
% \end{macro}
%
% \begin{macro}{\fao@oddstrip}
% \changes{v1.00}{2012/11/19}{Introduced the macro}
%   This is the strip on odd pages
%    \begin{macrocode}
\def\fao@oddstrip{%
  \if@altMargins
    \begin{picture}(0,0)%
      \put(530,-219){\raisebox{-0.717\Gm@layoutheight}{%
          \color{@bgcolor}\rule{0.6cm}{1.017\Gm@layoutheight}}}%
    \end{picture}%
  \else\if@altMarginsNarrow
    \begin{picture}(0,0)%
      \put(544,-219){\raisebox{-0.717\Gm@layoutheight}{%
          \color{@bgcolor}\rule{0.6cm}{1.017\Gm@layoutheight}}}%
    \end{picture}%
  \else
    \begin{picture}(0,0)%
      \put(530,-219){\raisebox{-0.717\Gm@layoutheight}{%
          \color{@bgcolor}\rule{0.6cm}{1.017\Gm@layoutheight}}}%
    \end{picture}%
  \fi\fi}
%    \end{macrocode}
%   
% \end{macro}
%
% \begin{macro}{\fao@evenstrip}
% \changes{v1.00}{2012/11/19}{Introduced the macro}
%   This is the strip on even pages
%    \begin{macrocode}
\def\fao@evenstrip{%
  \if@altMargins
    \begin{picture}(0,0)%
      \put(-49,-219){\raisebox{-0.717\Gm@layoutheight}{%
          \color{@bgcolor}\rule{0.6cm}{1.017\Gm@layoutheight}}}%
     \end{picture}%
   \else\if@altMarginsNarrow
     \begin{picture}(0,0)%
       \put(-35,-219){\raisebox{-0.717\Gm@layoutheight}{%
           \color{@bgcolor}\rule{0.6cm}{1.017\Gm@layoutheight}}}%
      \end{picture}%
    \else
      \begin{picture}(0,0)%
        \put(-63,-219){\raisebox{-0.717\Gm@layoutheight}{%
            \color{@bgcolor}\rule{0.6cm}{1.017\Gm@layoutheight}}}%
      \end{picture}%
  \fi\fi}
%    \end{macrocode}
%   
% \end{macro}
%
% \begin{macro}{standardpagestyle}
%   \changes{v0.4}{2011/08/28}{Changed colors}
%   \changes{v0.5}{2011/09/08}{Added strip}
%   \changes{v0.11}{2011/09/22}{Changed dimensions and moved the
%     folios} \changes{v0.12}{2011/09/28}{Used \cmd{\Gm@layoutheight}
%     and \cmd{\Gm@layoutwidth}} \changes{v0.24}{2011/11/16}{Changed
%     bleeds} \changes{v0.42}{2012/11/15}{Added altMargns}
%   \changes{v0.43}{2012/11/15}{Added altMargnsNarrow}
%   \changes{v1.00}{2012/11/19}{Changed color scheme}
%   \changes{v1.00}{2012/11/19}{Moved to standard strips}
%   \changes{v1.11}{2013/03/23}{Deletes section header on verso pages}
%   The pagestyle for ``normal'' pages
%    \begin{macrocode}
\fancypagestyle{standardpagestyle}{%
  \fancyhf{}%
  \fancyhfoffset{\z@}%
  \fancyhead[LO]{\fao@oddstrip}
  \fancyhead[LE]{\fao@evenstrip}%
  \fancyhead[RO]{%
    \color{black}\leftmark}%
  \fancyfoot[RO,LE]{\color{black}\thepage}%
  \renewcommand{\headrulewidth}{\z@}%
  \renewcommand{\footrulewidth}{\z@}}
%    \end{macrocode}
%   
% \end{macro}
%
%
% \begin{macro}{partpagestyle}
% \changes{v0.4}{2011/08/28}{Changed color on the right part}
% \changes{v0.4}{2011/08/28}{Changed margins}
% \changes{v0.5}{2011/09/08}{Changed size}
% \changes{v0.11}{2011/09/22}{Changed dimensions}
% \changes{v0.12}{2011/09/28}{Changed dimensions}
% \changes{v0.12}{2011/09/28}{Used \cmd{\Gm@layoutheight} and
% \cmd{\Gm@layoutwidth}} 
% \changes{v0.24}{2011/11/16}{Changed bleeds}
% \changes{v1.00}{2012/11/19}{Changed color scheme}
%   Special page style for parts:
%    \begin{macrocode}
\fancypagestyle{partpagestyle}{%
  \fancyhf{}%
  \fancyhfoffset{\z@}%
  \fancyhead[LO]{%
    \begin{picture}(0,0)%
      \put(-216,-174){\raisebox{-0.717\Gm@layoutheight}{%
          \color{@bgcolor!10}\rule{0.295\Gm@layoutwidth}{1.017\Gm@layoutheight}%
          \color{@bgcolor!7}\rule{0.728\Gm@layoutwidth}{1.017\Gm@layoutheight}}}%
    \end{picture}}%
  \fancyhead[LE]{%    
    \begin{picture}(0,0)%
      \put(-63,-174){\raisebox{-0.717\Gm@layoutheight}{%
          \color{@bgcolor!7}\rule{0.728\Gm@layoutwidth}{1.017\Gm@layoutheight}%
          \color{@bgcolor!10}\rule{0.295\Gm@layoutwidth}{1.017\Gm@layoutheight}}}%
    \end{picture}}
  \renewcommand{\headrulewidth}{\z@}%
  \renewcommand{\footrulewidth}{\z@}}
%    \end{macrocode}
%   
% \end{macro}
%
% \begin{macro}{indicatorfirstpagestyle}
% \changes{v0.4}{2011/08/28}{Changed colors}
% \changes{v0.4}{2011/08/28}{Changed margins}
% \changes{v0.5}{2011/09/08}{Changed size}
% \changes{v0.11}{2011/09/22}{Changed dimensions and moved the folios}
% \changes{v0.11}{2011/09/22}{Changed marks}
% \changes{v0.12}{2011/09/28}{Used \cmd{\Gm@layoutheight} and
% \cmd{\Gm@layoutwidth}} 
% \changes{v0.42}{2011/11/14}{Added check for altMargins}
% \changes{v0.43}{2011/11/15}{Added check for altMarginsNarrow}
% \changes{v1.00}{2012/11/19}{Changed color scheme}
% \changes{v1.04}{2012/11/26}{Corrected a bug}
%   \changes{v1.11}{2013/03/23}{Deletes section header on verso pages}
%   The style for the first indicator page (or all pages for
%   |indicatorA|): 
%    \begin{macrocode}
\fancypagestyle{indicatorfirstpagestyle}{%
  \fancyhf{}%
    \if@altMargins
    \fancyhead[LE]{%
      \begin{picture}(0,0)
        \put(-105,-219){\raisebox{-0.717\Gm@layoutheight}{%
            \color{white}\rule{0.6\Gm@layoutwidth}{1.017\Gm@layoutheight}%
            \color{@bgcolor!7}\rule{0.515\Gm@layoutwidth}{1.017\Gm@layoutheight}}}%
      \end{picture}\fao@evenstrip}%
    \fancyhead[LO]{%
      \begin{picture}(0,0)
        \put(-63,75){\raisebox{-0.717\Gm@layoutheight}{%
            \color{white}\rule{0.6\Gm@layoutwidth}{1.017\Gm@layoutheight}%
            \color{@bgcolor!7}\rule{0.515\Gm@layoutwidth}{1.017\Gm@layoutheight}}}%
      \end{picture}\fao@oddstrip}%
    \else\if@altMarginsNarrow
    \fancyhead[LE]{%
      \begin{picture}(0,0)
        \put(-91,-219){\raisebox{-0.717\Gm@layoutheight}{%
            \color{white}\rule{0.6\Gm@layoutwidth}{1.017\Gm@layoutheight}%
            \color{@bgcolor!7}\rule{0.515\Gm@layoutwidth}{1.017\Gm@layoutheight}}}%
      \end{picture}\fao@evenstrip}%
    \fancyhead[LO]{%
      \begin{picture}(0,0)
        \put(-63,75){\raisebox{-0.717\Gm@layoutheight}{%
            \color{white}\rule{0.6\Gm@layoutwidth}{1.017\Gm@layoutheight}%
            \color{@bgcolor!7}\rule{0.515\Gm@layoutwidth}{1.017\Gm@layoutheight}}}%
      \end{picture}\fao@oddstrip}%
    \else
    \fancyhead[LE]{%
      \begin{picture}(0,0)
        \put(-120,-219){\raisebox{-0.717\Gm@layoutheight}{%
            \color{white}\rule{0.6\Gm@layoutwidth}{1.017\Gm@layoutheight}%
            \color{@bgcolor!7}\rule{0.515\Gm@layoutwidth}{1.017\Gm@layoutheight}}}%
      \end{picture}\fao@evenstrip}%
    \fancyhead[LO]{%
      \begin{picture}(0,0)
        \put(-63,75){\raisebox{-0.717\Gm@layoutheight}{%
            \color{white}\rule{0.6\Gm@layoutwidth}{1.017\Gm@layoutheight}%
            \color{@bgcolor!7}\rule{0.515\Gm@layoutwidth}{1.017\Gm@layoutheight}}}%
      \end{picture}\fao@oddstrip}%
  \fi\fi
  \fancyhead[RO]{\color{black}\leftmark}%
  \renewcommand{\headrulewidth}{\z@}%
  \renewcommand{\footrulewidth}{\z@}%
  \fancyfoot[RO]{\color{black}\thepage}%
  \fancyfoot[LE]{\color{black}\thepage}}
%    \end{macrocode}
%   
% \end{macro}
%
% \begin{macro}{indicatorpagestyle}
% \changes{v0.4}{2011/08/28}{Changed colors}
% \changes{v0.5}{2011/09/08}{Changed size}
% \changes{v0.11}{2011/09/22}{Changed dimensions and moved the folios}
% \changes{v0.11}{2011/09/22}{Changed marks}
% \changes{v0.12}{2011/09/28}{Used \cmd{\Gm@layoutheight} and
% \cmd{\Gm@layoutwidth}} 
% \changes{v0.24}{2011/11/16}{Changed bleeds}
% \changes{v0.42}{2011/11/14}{Added check for altMargins}
% \changes{v0.43}{2011/11/15}{Added check for altMarginsNarrow}
% \changes{v1.00}{2012/11/19}{Changed color scheme}
% \changes{v1.00}{2012/11/19}{Moved to standard strips}
% \changes{v1.11}{2013/03/23}{Deletes section header on verso pages}
%   The style for the indicator pages: 
%    \begin{macrocode}
\fancypagestyle{indicatorpagestyle}{%
  \fancyhf{}%
  \fancyhfoffset{\z@}
  \if@altMargins
    \fancyhead[LE]{%
      \begin{picture}(0,0)%
        \put(-63,-219){\raisebox{-0.717\Gm@layoutheight}{%
            \color{@bgcolor!7}\rule{1.02\Gm@layoutwidth}{1.017\Gm@layoutheight}}}%
      \end{picture}\fao@evenstrip}%
    \fancyhead[LO]{%
      \begin{picture}(0,0)%
        \put(-63,-219){\raisebox{-0.717\Gm@layoutheight}{%
            \color{@bgcolor!7}\rule{1.02\Gm@layoutwidth}{1.017\Gm@layoutheight}}}%
      \end{picture}\fao@oddstrip}%
  \else\if@altMarginsNarrow
    \fancyhead[LE]{%
      \begin{picture}(0,0)%
        \put(-49,-219){\raisebox{-0.717\Gm@layoutheight}{%
            \color{@bgcolor!7}\rule{1.02\Gm@layoutwidth}{1.017\Gm@layoutheight}}}%
      \end{picture}\fao@evenstrip}%
    \fancyhead[LO]{%
      \begin{picture}(0,0)%
        \put(-49,-219){\raisebox{-0.717\Gm@layoutheight}{%
            \color{@bgcolor!7}\rule{1.02\Gm@layoutwidth}{1.017\Gm@layoutheight}}}%
      \end{picture}\fao@oddstrip}%
  \else
    \fancyhead[LE]{%
      \begin{picture}(0,0)%
        \put(-63,-219){\raisebox{-0.717\Gm@layoutheight}{%
            \color{@bgcolor!7}\rule{1.02\Gm@layoutwidth}{1.017\Gm@layoutheight}}}%
      \end{picture}\fao@evenstrip}%
    \fancyhead[LO]{%
      \begin{picture}(0,0)%
        \put(-63,-219){\raisebox{-0.717\Gm@layoutheight}{%
            \color{@bgcolor!7}\rule{1.02\Gm@layoutwidth}{1.017\Gm@layoutheight}}}%
      \end{picture}\fao@oddstrip}%
  \fi\fi
  \fancyhead[RO]{\color{black}\leftmark}%
  \renewcommand{\headrulewidth}{\z@}%
  \renewcommand{\footrulewidth}{\z@}%
  \fancyfoot[RO,LE]{\color{black}\thepage}}
%    \end{macrocode}
%   
% \end{macro}
%
% \begin{macro}{indicatorDpagestyle}
% \changes{v1.00}{2012/11/18}{Added macro}
% \changes{v1.00}{2012/11/19}{Changed color scheme}
% \changes{v1.00}{2012/11/19}{Moved to standard strips}
% \changes{v1.11}{2013/03/23}{Deletes section header on verso pages}
%   This is the style for the first page of indicatorD.  The page is
%   divided vertically instead of horizontally.
%    \begin{macrocode}
\fancypagestyle{indicatorDpagestyle}{%
  \fancyhf{}%
  \fancyhfoffset{\z@}
  \if@altMargins
    \fancyhead[LE]{%
      \begin{picture}(0,0)%
        \put(-49,-219){\raisebox{-0.717\Gm@layoutheight}{%
            \color{@bgcolor!7}\rule{1.02\Gm@layoutwidth}{0.5\Gm@layoutheight}}}%
      \end{picture}\fao@evenstrip}%
    \fancyhead[LO]{%
      \begin{picture}(0,0)%
        \put(-63,-219){\raisebox{-0.717\Gm@layoutheight}{%
            \color{@bgcolor!7}\rule{1.02\Gm@layoutwidth}{0.5\Gm@layoutheight}}}%
      \end{picture}\fao@oddstrip}%
  \else\if@altMarginsNarrow
    \fancyhead[LE]{%
      \begin{picture}(0,0)%
        \put(-35,-219){\raisebox{-0.717\Gm@layoutheight}{%
            \color{@bgcolor!7}\rule{1.02\Gm@layoutwidth}{0.5\Gm@layoutheight}}}%
      \end{picture}\fao@evenstrip}%
    \fancyhead[LO]{%
      \begin{picture}(0,0)%
        \put(-49,-219){\raisebox{-0.717\Gm@layoutheight}{%
            \color{@bgcolor!7}\rule{1.02\Gm@layoutwidth}{0.5\Gm@layoutheight}}}%
      \end{picture}\fao@oddstrip}%
  \else
    \fancyhead[LE]{%
      \begin{picture}(0,0)%
        \put(-63,-219){\raisebox{-0.717\Gm@layoutheight}{%
            \color{@bgcolor!7}\rule{1.02\Gm@layoutwidth}{0.5\Gm@layoutheight}}}%
      \end{picture}\fao@evenstrip}%
    \fancyhead[LO]{%
      \begin{picture}(0,0)%
        \put(-63,-219){\raisebox{-0.717\Gm@layoutheight}{%
            \color{@bgcolor!7}\rule{1.02\Gm@layoutwidth}{0.5\Gm@layoutheight}}}%
      \end{picture}\fao@oddstrip}%
  \fi\fi
  \fancyhead[RO]{\color{black}\leftmark}%
  \renewcommand{\headrulewidth}{\z@}%
  \renewcommand{\footrulewidth}{\z@}%
  \fancyfoot[RO]{\color{black}\thepage}%
  \fancyfoot[LE]{\color{black}\thepage}}
%    \end{macrocode}
% \end{macro}
%
% \begin{macro}{tablepage}
% \changes{v0.2}{2011/08/23}{Added macro}
% \changes{v0.5}{2011/09/07}{Renamed macro and made color selectable}
% \changes{v0.5}{2011/09/08}{Changed size}
% \changes{v0.9}{2011/09/14}{Moved color out}
% \changes{v0.12}{2011/09/28}{Used \cmd{\Gm@layoutheight} and
% \cmd{\Gm@layoutwidth}} 
% \changes{v0.24}{2011/11/16}{Added coloring}
% \changes{v0.31}{2011/11/26}{Added page numbers}
% \changes{v0.42}{2012/11/14}{Added check for altMargns}
% \changes{v0.43}{2012/11/14}{Added check for altMargnsNarrow}
% \changes{v1.00}{2012/11/19}{Changed color scheme}
% \changes{v1.02}{2012/11/21}{Integrated changes in margins}
%   Pages for tables
%    \begin{macrocode}
\fancypagestyle{tablepage}{%
  \fancyhf{}%
  \fancyhfoffset{\z@}
  \if@altMargins
  \fancyhead[LE]{%
    \begin{picture}(0,0)%
      \put(-48,-255){\raisebox{-0.717\Gm@layoutheight}{%
          \color{@tablebg!10}\rule{1.02\Gm@layoutwidth}{1.017\Gm@layoutheight}}}%
    \end{picture}}%
  \fancyhead[LO]{%
    \begin{picture}(0,0)%
      \put(-62,-255){\raisebox{-0.717\Gm@layoutheight}{%
          \color{@tablebg!10}\rule{1.02\Gm@layoutwidth}{1.017\Gm@layoutheight}}}%
    \end{picture}}%
  \else\if@altMarginsNarrow
  \fancyhead[LE]{%
    \begin{picture}(0,0)%
      \put(-34,-255){\raisebox{-0.717\Gm@layoutheight}{%
          \color{@tablebg!10}\rule{1.02\Gm@layoutwidth}{1.017\Gm@layoutheight}}}%
    \end{picture}}%
  \fancyhead[LO]{%
    \begin{picture}(0,0)%
      \put(-48,-255){\raisebox{-0.717\Gm@layoutheight}{%
          \color{@tablebg!10}\rule{1.02\Gm@layoutwidth}{1.017\Gm@layoutheight}}}%
    \end{picture}}%
  \else
  \fancyhead[LE]{%
    \begin{picture}(0,0)%
      \put(-20,-260){\raisebox{-0.717\Gm@layoutheight}{%
          \color{@tablebg!10}\rule{1.02\Gm@layoutwidth}{1.017\Gm@layoutheight}}}%
    \end{picture}}%
  \fancyhead[LO]{%
    \begin{picture}(0,0)%
      \put(-20,-260){\raisebox{-0.717\Gm@layoutheight}{%
          \color{@tablebg!10}\rule{1.02\Gm@layoutwidth}{1.017\Gm@layoutheight}}}%
    \end{picture}}%
  \setlength{\footskip}{15pt}%
  \fi\fi
  \renewcommand{\headrulewidth}{\z@}%
  \renewcommand{\footrulewidth}{\z@}%
  \fancyfoot[RO,LE]{\thepage}}
%    \end{macrocode}
% \end{macro}
%
% \begin{macro}{conceptspagestyle}
% \changes{v1.05}{2012/12/27}{Added macro}
%   The pagestyle for ``concept'' pages
%    \begin{macrocode}
\fancypagestyle{conceptpagestyle}{%
  \fancyhf{}%
  \fancyhfoffset{\z@}%
  \fancyhead[LE]{\color{black}}%
  \fancyhead[RO]{%
    \color{black}\leftmark}%
  \fancyfoot[RO,LE]{\color{black}\thepage}%
  \renewcommand{\headrulewidth}{\z@}%
  \renewcommand{\footrulewidth}{\z@}}
%    \end{macrocode}
%      
% \end{macro}
% 
%
%\subsection{Sectioning}
%\label{sec:sectioning}
%
% \begin{macro}{\if@mainmatter}
% \changes{v0.4}{2011/08/28}{Added macro}
%   This is used to check whether we are at main matter
%    \begin{macrocode}
\newif\if@mainmatter
%    \end{macrocode}
%   
% \end{macro}
%
% \begin{macro}{\frontmatter}
% \changes{v0.4}{2011/08/28}{Added macro}
% \changes{v0.7}{2011/09/11}{Added change of geometry}
% \changes{v0.29}{2011/11/25}{Added \cmd{\clerdoublepage}}
%   We want Roman numbers for front matter:
%    \begin{macrocode}
\def\frontmatter{\cleardoublepage
  \pagenumbering{roman}\onecolumn\@mainmatterfalse}
%    \end{macrocode}
%   
% \end{macro}
% \begin{macro}{\mainmatter}
% \changes{v0.4}{2011/08/28}{Added macro}
% \changes{v0.7}{2011/09/11}{Added change of geometry}
% \changes{v0.12}{2011/09/28}{Added \cmd{\cleardoublepage}}
%   We want Arabic numbers for main matter:
%    \begin{macrocode}
\def\mainmatter{\cleardoublepage\pagenumbering{arabic}\twocolumn
  \@mainmattertrue}
%    \end{macrocode}
%   
% \end{macro}
%
% \begin{macro}{\tocdepth}
% \changes{v0.15}{2011/10/12}{Redefined}
%   Only sections and up are allowed in TOC:
%    \begin{macrocode}
\setcounter{tocdepth}{1}
%    \end{macrocode}   
% \end{macro}
% \begin{macro}{\secnumdepth}
%   Only the parts are numbered in out setup:
%    \begin{macrocode}
\setcounter{secnumdepth}{-1}
%    \end{macrocode}  
% \end{macro}
% \begin{macro}{\thepart}
%   And the parts are numbered using Arabic numbers:
%    \begin{macrocode}
\renewcommand \thepart {\@arabic\c@part}
%    \end{macrocode}
% \end{macro}
%
% \begin{macro}{\c@fao@partnum}
% \changes{v0.36}{2011/12/19}{Introduced macro}
%   To draw the blobs in part color in the proper position, we need to
%   associate them with parts.   However, some parts are numbered,
%   some are note.  The macro |\fao@partnum| keeps the current part
%   number counted continuously from the beginning to end.
%    \begin{macrocode}
\newcounter{fao@partnum}
\setcounter{fao@partnum}{0}
%    \end{macrocode}
% \end{macro}
%
% \begin{macro}{\fao@currentpartnum}
% \changes{v0.36}{2011/12/19}{Introduced macro}
%   The current value of |\fao@partnum| used in TOC:
%    \begin{macrocode}
\def\fao@currentpartnum{0}
%    \end{macrocode}
%   
% \end{macro}
%
% \begin{macro}{\part}
% \changes{v0.2}{2011/08/23}{Added selectcolor}
% \changes{v0.4}{2011/08/30}{Added rowcolors}
% \changes{v0.32}{2011/12/05}{Added color switch}
% \changes{v0.36}{2011/12/19}{Added faopartnum}
% \changes{v0.36}{2011/12/29}{Fixed bug with \cmd{\cleardoublepage}}
% \changes{v0.41}{2012/11/14}{Added check for altMargins}
% \changes{v0.42}{2012/11/15}{Added check for altMarginsNarrow}
%   This is basicale the same as in~\cite{classes}, with the change
%   that we want to use our own page style.
%   it.  
%    \begin{macrocode}
\renewcommand\part{%
  \cleardoublepage
  \addtocontents{toc}{\string\colorlet{@bgcolor}[cmyk]{\fao@color@string}}%
  \stepcounter{fao@partnum}%
%    \end{macrocode}
%    Now we set \cmd{\fao@blobposition} and \cmd{\fao@blobheight} to
%    the correct values---if defined
%    \begin{macrocode}
  \expandafter\ifx\csname fao@blobstart\thefao@partnum\endcsname\relax
     \setlength{\fao@blobposition}{-100em}%
  \else
     \setlength{\fao@blobposition}{\csname
       fao@blobstart\thefao@partnum\endcsname}%
  \fi
  \typeout{DEBUG: position \the\fao@blobposition}%
  \setlength{\fao@blobheight}{-\fao@blobposition}%
  \expandafter\ifx\csname fao@blobend\thefao@partnum\endcsname\relax
     \setlength{\fao@blobheight}{0em}%
  \else
     \addtolength{\fao@blobheight}{\csname
       fao@blobend\thefao@partnum\endcsname}%
  \fi
  \typeout{DEBUG: height \the\fao@blobheight}%
  \addtocontents{toc}{%
    \string\gdef\string\fao@currentpartnum{\thefao@partnum}}%
  \ifprint
    \if@altMargins
     \newgeometry{layout=a4paper,left=0.5\textwidth,right=2cm,
      layouthoffset=5mm,layoutvoffset=5mm}%
    \else\if@altMarginsNarrow
     \newgeometry{layout=a4paper,left=0.5\textwidth,right=2cm,
      layouthoffset=5mm,layoutvoffset=5mm}%
    \else
     \newgeometry{layout=a4paper,left=0.5\textwidth,right=2cm,
      layouthoffset=1cm,layoutvoffset=1cm}%
    \fi\fi
  \else
  \newgeometry{layout=a4paper,left=0.5\textwidth,right=2cm}%
  \fi
  \selectcolor
\rowcolors{2}{@bgcolor!10}{}%
  \pagestyle{partpagestyle}%
  \if@twocolumn
    \onecolumn
  \fi
  \secdef\@part\@spart}
%    \end{macrocode}
% \end{macro}
%
% \begin{macro}{\@part}
% \changes{v0.5}{2011/09/08}{The transparency of the part number is
% now provided by TikZ}
% \changes{v0.18}{2011/11/02}{Moved the number}
% \changes{v1.00}{2012/11/19}{Changed color scheme}
% \changes{v1.02}{2012/11/26}{Increased font}
%   This is the actual part making macro.
%    \begin{macrocode}
\def\@part[#1]#2{%
      \refstepcounter{part}%
      \addcontentsline{toc}{part}{\thepart\hspace{1em}#1}%
    \markboth{\MakeUppercase{#1}}{\MakeUppercase{\partname}~\thepart}%
    {\interlinepenalty \@M
     \normalfont\null\hspace{-70mm}%
     \fontsize{24\p@}{32\p@}\selectfont \bfseries
     \raisebox{25\p@}{\color{@bgcolor!70}\MakeUppercase{\partname}}%
     \hspace{35\p@}%
     \raisebox{-180\p@}{\fontsize{250\p@}{200\p@}\selectfont \bfseries
       \color{@bgcolor}\begin{tikzpicture}%
         \node[opacity=0.25,inner sep=\z@]{\thepart};%
       \end{tikzpicture}}%
     \parbox[t]{0.8\textwidth}{\raggedright
       \color{@bgcolor}#2}\par}\vspace{100\p@}\bgroup
 \large}
%    \end{macrocode}
% \end{macro}
% 
% \begin{macro}{\@spart}
% \changes{v0.26}{2011/11/22}{Refedined}
% \changes{v1.02}{2012/11/26}{Increased font}
%   This produces \emph{unnumbered} parts:
%    \begin{macrocode}
\def\@spart#1{%
     \phantomsection
      \addcontentsline{toc}{spart}{#1}%
    \markboth{#1}{#1}%
    {\interlinepenalty \@M
     \normalfont\null\hspace{-70mm}%
     \fontsize{24\p@}{32\p@}\selectfont \bfseries
     \raisebox{25\p@}{\color{@bgcolor!70}\phantom{\MakeUppercase{\partname}}}%
     \hspace{35\p@}%
     \raisebox{-180\p@}{\fontsize{250\p@}{200\p@}\selectfont \bfseries
       \color{@bgcolor!20}\begin{tikzpicture}%
         \node[opacity=0.25,inner sep=\z@]{\phantom{\thepart}};%
       \end{tikzpicture}}%
     \parbox[t]{0.8\textwidth}{\raggedright
       \color{@bgcolor}#1}\par}\vspace{100\p@}\bgroup
 \large}
%    \end{macrocode}   
% \end{macro}
%
% \begin{macro}{\EndPartIntro}
% \changes{v0.18}{2011/11/06}{Changed style of empty pages}
% \changes{v1.02}{2012/11/26}{Increased font}
%   This command switches the  special formatting of part pages back:
%    \begin{macrocode}
\def\EndPartIntro{%
  \egroup\twocolumn
  \pagecolor{white}\color{black}%
  \loadgeometry{standard}%
  \pagestyle{standardpagestyle}}
%    \end{macrocode}
%   
% \end{macro}
%
% \begin{macro}{\chaptermark}
% \changes{v1.02}{2012/11/26}{Added mark}
% We use chapter mark in headers
%    \begin{macrocode}
\renewcommand{\chaptermark}[1]{\markright{#1}}
%    \end{macrocode}
% \end{macro}
% \begin{macro}{\sectionmark}
%   We do not use section info in headers:
%    \begin{macrocode}
\renewcommand{\sectionmark}[1]{}
%    \end{macrocode}
% \end{macro}
%
% \begin{macro}{\section}
% \changes{v0.18}{2011/11/02}{Changed vertical spacing}
% \changes{v0.19}{2011/11/04}{Changed vertical spacing and fonts}
%   We changed the standard \cmd{section} defaults:
%    \begin{macrocode}
\renewcommand\section{\@startsection {section}{1}{\z@}%
                                   {\z@ plus 6\p@}%
                                   {\parskip}%
                                   {\normalfont\large\bfseries}}
%    \end{macrocode}
%   
% \end{macro}
%
% \begin{macro}{\subsection}
% \changes{v0.19}{2011/11/04}{Changed vertical spacing and fonts}
%   We changed the standard \cmd{subsection} defaults:
%    \begin{macrocode}
\renewcommand\subsection{\@startsection {subsection}{2}{\z@}%
                                   {\z@ plus 6\p@}%
                                   {\parskip}%
                                   {\normalfont\large\itshape}}
%    \end{macrocode}
%   
% \end{macro}
%
%\subsection{Setting Colors}
%\label{sec:colors}
%
% \begin{macro}{\fao@color@string}
% \changes{v0.32}{2011/12/05}{Added the macro}
%   This is the command that remembers the present color for TOC
%    \begin{macrocode}
\def\fao@color@string{0,0,0}
%    \end{macrocode}
%   
% \end{macro}
%
% \begin{macro}{@bgcolor@next}
% \changes{v0.2}{2011/08/23}{Added the macro}
%   We store the next background color in |@bgcolor@next|.
%  We store the next heading background in |@tableheadcolor@next|.
% \begin{macro}{\setbgcolor}
% \changes{v0.2}{2011/08/23}{Made the color change delayed}
% \changes{v0.7}{2011/09/11}{Wrote the current color into the toc}
% \changes{v0.32}{2011/12/05}{Changed the way the current color is stored}
%   The command |\setbgcolor| selects the next background color:
%    \begin{macrocode}
\def\setbgcolor#1{\colorlet{@bgcolor@next}[cmyk]{#1}%
  \addtocontents{toc}{\string\colorlet{@bgcolor}[cmyk]{#1}}%
  \gdef\fao@color@string{#1}}
\setbgcolor{red}
%    \end{macrocode}
% \end{macro}   
% \end{macro}
%
%  The key-value interface for the same command:
%    \begin{macrocode}
\define@key{fao}{bgcolor}{\setbgcolor{#1}}
%    \end{macrocode}
%  
% And for separate setting of |@tableheadcolor|
%    \begin{macrocode}
\define@key{fao}{tableheadcolor}{\colorlet{@tableheadcolor}[cmyk]{#1}}
%    \end{macrocode}
%
% \begin{macro}{@bgcolor}
% \changes{v0.2}{2011/08/23}{Made the color change delayed}
% \changes{v0.8}{2011/09/12}{Added change in @tableheadcolor}
% The current color is in the macro |@bgcolor|.
% \begin{macro}{@tableheadcolor@next}
% \changes{v0.8}{2011/09/12}{Added the macro}
% \begin{macro}{\selectcolor}
%   This command makes the actual color change:
%    \begin{macrocode}
\def\selectcolor{\colorlet{@bgcolor}{@bgcolor@next}%
  \colorlet{@tableheadcolor}{@bgcolor!30}}
\selectcolor
%    \end{macrocode}
%   
% \end{macro}
% \end{macro}
% \end{macro}
% 
% \begin{macro}{@tablebg}
% \changes{v0.2}{2011/09/07}{Added macro}
%   The color for table pages
%    \begin{macrocode}
\define@key{fao}{tablebg}{\colorlet{@tablebg}[cmyk]{#1}}
%    \end{macrocode}
%   
% \end{macro}
%
%\subsection{Indicator Pages}
%\label{sec:indicator}
%
% \begin{macro}{\indicatorText}
%   Normally (outside of indicator) this should produce an error
%   message:
%    \begin{macrocode}
\newenvironment{indicatorText}{\PackageError{faoyearbook}{%
  Environment `indicatorText' in a wrong place!}{%
This environment must be inside indicator pages.}}{}
%    \end{macrocode}
% \end{macro}
% \begin{macro}{\indicatorGraphics}
%   Normally (outside of indicator) this should produce an error
%   message:
%    \begin{macrocode}
\newenvironment{indicatorGraphics}{\PackageError{faoyearbook}{%
  Environment `indicatorGraphics' in a wrong place!}{%
This environment must be inside indicator pages.}}{}
%    \end{macrocode}
% \end{macro}
%
% \begin{macro}{\indicatorA}
%   Version~A.  We use parallel processing.  Most of the code of the
%   here follows~\cite{Eckermann:Parallel} with the patches
%   from~\cite{Oberdiek:Pdfcolparallel}.
%    \begin{macrocode}
\def\indicatorA{%
%    \end{macrocode}
% First, we switch to the next recto page::
%    \begin{macrocode}
  \clearpage\ifodd\c@page\hbox{}\newpage\fi
%    \end{macrocode}
% Next, we need to change to one column and remember whether to
% switch back to one column mode.  We also now are ready for the new
% page style
%    \begin{macrocode}
  \if@twocolumn
    \onecolumn
    \@tempswatrue
  \else
    \@tempswafalse
  \fi
  \pagestyle{indicatorfirstpagestyle}%
%    \end{macrocode}
% 
% Now we switch on |Parallel| environment.  We want to redefine the
% command |\ParallelParOnePage| (more on this below), so we do this
% inside a group to keep the change private:
%    \begin{macrocode}
  \bgroup
  \def\ParallelParOnePage{\FAO@ParallelParOnePage}%
  \begin{Parallel}{0.45\textwidth}{0.45\textwidth}%
%    \end{macrocode}
%
%  We define |\indicatorText| and |\indicatorGraphics| inside the
%  |\indicator| so they are local to the group.
%    \begin{macrocode}
\def\indicatorText{\ParallelLText\bgroup\columnwidth=\ParallelLWidth
  \strut}%
\def\endindicatorText{\egroup}
\def\indicatorGraphics{\ParallelRText\bgroup\strut
  \columnwidth=\ParallelRWidth}%
\def\endindicatorGraphics{\egroup}}%
%    \end{macrocode}
% \end{macro}
% 
% \begin{macro}{\endindicatorA}
% \changes{v0.4}{2011/08/30}{Color adjustment}
% \changes{v0.5}{2011/09/08}{Added new page style}
%   Wrapping up
%    \begin{macrocode}
\def\endindicatorA{\end{Parallel}\egroup
%    \end{macrocode}
% Switching to two column if needed:
%    \begin{macrocode}
    \if@tempswa\twocolumn\else\onecolumn\fi
    \clearpage\color{black}%
    \global\let\default@color\current@color\normalcolor
    \loadgeometry{standard}%
    \pagestyle{standardpagestyle}}
%    \end{macrocode}
% \end{macro}
%
% \begin{macro}{\FAO@ParallelParOnePage}
%   The package \progname{parallel} tries to achieve aligned lines in
%   two columns.  The way it does it is the following: it takes one
%   line from each column and puts them into boxes, then again and
%   again.  Which is fine, but we explicitly do \emph{not} want this
%   effect, otherwise the lines in the text part look really funny.
%   So we redefine the routine |\ParallelParOnePage| to use larger
%   boxes.  We also need the patches from \progname{pdfcolparallel} to
%   keep the color stack.
%
%   So we change the lines |\vsplit BOX to\dp\strutbox| to 
%   |\vsplit BOX to\@tempdima| where |\@tempdima| is the page height
%   minus one basline.  We also use |\vtop| instead of |\vbox| to
%   achieve the proper alignment.
%    \begin{macrocode}
\def\FAO@ParallelParOnePage{%
  \ifnum\ParallelBoolVar=\@ne
    \par
    \begingroup
      \leftmargin=\z@
      \rightmargin=\z@
      \vbadness=10000 %
      \vfuzz=3ex %
      \splittopskip=\z@skip
      \@tempdima=\@colroom
      \advance\@tempdima by -\baselineskip
      \loop
        \ifnum\ParallelBoolVar=\@ne
          \noindent
          \hbox to\textwidth{%
            \hskip\ParallelLeftMargin
            \hbox to\ParallelTextWidth{%
              \ifvoid\ParallelLBox
                \hskip\ParallelLWidth
              \else
                \pcp@SetCurrent{Left}%
                \ParallelWhichBox=\z@
                \vtop{%
                  \setbox\ParallelBoxVar
                      =\vsplit\ParallelLBox to\@tempdima
                  \unvbox\ParallelBoxVar
                }%
              \fi
              \ifnum\ParallelBoolMid=\@ne
                \hskip\ParallelMainMidSkip
                \begingroup
                  \pcp@RuleBetweenColor
                  \vrule
                \endgroup
              \else
                \hss
              \fi
              \hss
              \ifvoid\ParallelRBox
                \hskip\ParallelRWidth
              \else
                \pcp@SetCurrent{Right}%
                \ParallelWhichBox=\@ne
                \vtop{%
                  \setbox\ParallelBoxVar
                      =\vsplit\ParallelRBox to\@tempdima
                      \vfill
                  \unvbox\ParallelBoxVar
                }%
              \fi
            }%
          }%
          \ifvoid\ParallelLBox
            \ifvoid\ParallelRBox
              \global\ParallelBoolVar=\z@
            \fi
          \fi%
        \fi%
      \ifnum\ParallelBoolVar=\@ne
        \penalty\interlinepenalty
      \repeat
      \par
    \endgroup
    \pcp@SetCurrent{}%
  \fi}
%    \end{macrocode}
% \end{macro}
%
% \begin{macro}{\indicatorB}
%   Here we have the first page with text and graphics, but
%   consecutive pages have graphics only.  This follows |\indicatorA|:
%    \begin{macrocode}
\def\indicatorB{%
  \clearpage\ifodd\c@page\hbox{}\clearpage\fi
%    \end{macrocode}
% Next, we need to change to two column and remember whether to
% switch back.  We also now are ready for the new
% page style
%    \begin{macrocode}
  \if@twocolumn
    \@tempswatrue
  \else
    \twocolumn
    \@tempswafalse
  \fi
  \pagestyle{indicatorpagestyle}%
%    \end{macrocode}
%   Then we define |indicatorText| and |indicatorGraphics|.  Note that
%   |indicatorText| starts a spread and introduces special page style: 
%    \begin{macrocode}
  \bgroup
  \def\indicatorText{\bgroup\twocolumn
    \clearpage\ifodd\c@page\hbox{}\clearpage\fi
    \thispagestyle{indicatorfirstpagestyle}}%
  \def\endindicatorText{\newpage\egroup}
  \def\indicatorGraphics{\bgroup}%
  \def\endindicatorGraphics{\egroup}}%
%    \end{macrocode}
% \end{macro}
%
% \begin{macro}{\endindicatorB}
% \changes{v0.4}{2011/08/30}{Color adjustment}
% \changes{v0.5}{2011/09/08}{Added new page style}
%   Wrapping up |indicatorB|:
%    \begin{macrocode}
\def\endindicatorB{\egroup\color{black}%
  \global\let\default@color\current@color\normalcolor
  \if@tempswa\twocolumn\else\onecolumn\fi
  \clearpage\loadgeometry{standard}%
  \pagestyle{standardpagestyle}}
%    \end{macrocode}
%   
% \end{macro}
%
%
% \begin{macro}{\indicatorC}
%   This follows |\indicatorB|:
%    \begin{macrocode}
\def\indicatorC{%
  \clearpage\ifodd\c@page\hbox{}\clearpage\fi
%    \end{macrocode}
% Next, we need to change to two column and remember whether to
% switch back.  We also now are ready for the new
% page style
%    \begin{macrocode}
  \if@twocolumn
    \@tempswatrue
  \else
    \twocolumn
    \@tempswafalse
  \fi
  \pagestyle{indicatorpagestyle}%
%    \end{macrocode}
%   Then we define |indicatorText| and |indicatorGraphics|:
%    \begin{macrocode}
  \bgroup
  \def\indicatorText{\PackageError{faoyearbook}{%
    Bad place for indicatorText}{%
  On indicatorC pages only graphics is allowed}}%
  \def\endindicatorText{\newpage\egroup}
  \def\indicatorGraphics{\bgroup}%
  \def\endindicatorGraphics{\egroup}}%
%    \end{macrocode}
% \end{macro}
%
% \begin{macro}{\endindicatorC}
% \changes{v0.4}{2011/08/30}{Color adjustment}
% \changes{v0.5}{2011/09/08}{Added new page style}
%   Wrapping up |indicatorC|:
%    \begin{macrocode}
\def\endindicatorC{\egroup\color{black}%
  \global\let\default@color\current@color\normalcolor
  \if@tempswa\twocolumn\else\onecolumn\fi
  \clearpage\loadgeometry{standard}%
  \pagestyle{standardpagestyle}}
%    \end{macrocode}
%   
% \end{macro}
%
% \begin{macro}{\indicatorD}
% \changes{v1.00}{2012/11/18}{Added macro}
%   This the indicator for horizontally divided graphics:
%    \begin{macrocode}
\def\indicatorD{%
  \clearpage\ifodd\c@page\hbox{}\clearpage\fi
%    \end{macrocode}
% Next, we need to change to one column and remember whether to
% switch back.  We also now are ready for the new
% page style
%    \begin{macrocode}
  \if@twocolumn
    \@tempswatrue
    \onecolumn
  \else
    \@tempswafalse
  \fi
  \thispagestyle{indicatorDpagestyle}%
  \pagestyle{indicatorpagestyle}%
%   Then we define |indicatorText| and |indicatorGraphics|:
%    \begin{macrocode}
  \bgroup
  \def\indicatorText{\begin{minipage}[t][0.5\textheight]{\textwidth}%
      \begin{multicols}{2}}%      
  \def\endindicatorText{\end{multicols}\end{minipage}}
  \def\indicatorGraphics{\bgroup}%
  \def\endindicatorGraphics{\egroup}}%
%    \end{macrocode}
%   
% \end{macro}
%
% \begin{macro}{\endindicatorD}
% \changes{v1.00}{2012/11/18}{Added up}
%   Wrapping up |indicatorD|:
%    \begin{macrocode}
\def\endindicatorD{\egroup\color{black}%
  \global\let\default@color\current@color\normalcolor
  \if@tempswa\twocolumn\else\onecolumn\fi
  \clearpage\loadgeometry{standard}%
  \pagestyle{standardpagestyle}}
%    \end{macrocode}
% \end{macro}
% 
% \begin{macro}{\indicator}
% \changes{v0.2}{2011/08/23}{Added macro}
%   An alias for |indicatorB| is just |indicator|:
%    \begin{macrocode}
\let\indicator\indicatorB
\let\endindicator\endindicatorB
%    \end{macrocode}
%   
% \end{macro}
%
% \begin{macro}{\indicatorPage}
% \changes{v0.13}{2011/10/03}{Added macro}
%   Another alias for |indicatorB| is |indicatorPage|:
%    \begin{macrocode}
\let\indicatorPage\indicatorB
\let\endindicatorPage\endindicatorB
%    \end{macrocode}
%   
% \end{macro}
%
%
%
%\subsection{Floats}
%\label{sec:floats}
%
%
% \begin{macro}{\float@caption}
% \changes{v0.4}{2011/08/29}{Added macro}
%   Package \progname{float} defines |\float@caption|, and
%   \progname{hyperref} tries to redefine it, which sometimes fails.
%   Since we use \progname{caption} anyway, we can safely disable this
%   macro:
%    \begin{macrocode}
\let\float@caption\@undefined
%    \end{macrocode}
%   
% \end{macro}
%
% Since we intend to use \progname{babel} in the future, we do not
% hard code the names for the floats.  Here we provide the English
% variants.
% \begin{macro}{\listchartname}
%   List of charts:
%    \begin{macrocode}
\newcommand\listchartname{List of Charts}
%    \end{macrocode}
% \end{macro}
% \begin{macro}{\listmapname}
%   List of maps:
%    \begin{macrocode}
\newcommand\listmapname{List of Maps}
%    \end{macrocode}
% \end{macro}
% \begin{macro}{\chartname}
%   A chart:
%    \begin{macrocode}
\newcommand\chartname{Chart}
%    \end{macrocode}
% \end{macro}
% \begin{macro}{\minitabname}
%   A chart:
%    \begin{macrocode}
\newcommand\minitabname{Mini Table}
%    \end{macrocode}
% \end{macro}
% \begin{macro}{\mapname}
%   A map:
%    \begin{macrocode}
\newcommand\mapname{Map}
%    \end{macrocode}
% \end{macro}
% 
% We introduce two new floats:
%    \begin{macrocode}
\newfloat{chart}{H}{loc}
\floatname{chart}{\chartname}
\newfloat{minitab}{H}{loc}
\floatname{minitab}{\minitabname}
\newfloat{map}{H}{lom}
\floatname{map}{\mapname}
%    \end{macrocode}
% 
% \begin{macro}{\listofcharts}
% \changes{v1.06}{2012/12/05}{Added hyperlinks}
% \changes{v1.10}{2013/02/26}{Added protect}
%   The list of charts:
%    \begin{macrocode}
\def\listofcharts{\listof{chart}{\protect\hypertarget{list:chart}{\listchartname}}}
%    \end{macrocode}   
% \end{macro}
% \begin{macro}{\listofmaps}
% \changes{v1.06}{2012/12/05}{Added hyperlinks}
% \changes{v1.10}{2013/02/26}{Added protect}
%   The list of maps:
%    \begin{macrocode}
\def\listofmaps{\listof{map}{\protect\hypertarget{list:map}{\listmapname}}}
%    \end{macrocode}   
% \end{macro}
% \begin{macro}{\listoftables}
% \changes{v1.06}{2012/12/05}{Added macro}
% \changes{v1.10}{2013/02/26}{Added protect}
%   The list of tables:
%    \begin{macrocode}
\def\listoftables{\listof{table}{\protect\hypertarget{list:table}{\listtablename}}}
%    \end{macrocode}   
% \end{macro}
% 
% \begin{macro}{\@floatwidth}
% \changes{v0.2}{2011/08/25}{Added macro}
%   The width of the floatbox:
%    \begin{macrocode}
\define@key{fao}{floatwidth}{\def\@floatwidth{#1}}
\faoset{floatwidth=\columnwidth}
%    \end{macrocode}
% \end{macro}
% \begin{macro}{\@floatheight}
% \changes{v0.2}{2011/08/25}{Added macro}
%   The height of the floatbox:
%    \begin{macrocode}
\define@key{fao}{floatheight}{\def\@floatheight{#1}}
\faoset{floatheight=0.45\textheight}
%    \end{macrocode}
% \end{macro}
% \begin{macro}{\ifKV@fao@fixedfloatheight}
% \changes{v0.2}{2011/08/25}{Added macro}
%   Whether to use fixed float height
%    \begin{macrocode}
\define@boolkey{fao}{fixedfloatheight}{}
\faoset{fixedfloatheight=false}
%    \end{macrocode}   
% \end{macro}
%
% \begin{macro}{\@beginfloat@hook}
% \changes{v0.2}{2011/08/25}{Added macro}
%   This is done at the beginning of a float:
%    \begin{macrocode}
\def\@beginfloat@hook{}
%    \end{macrocode}   
% \end{macro}
% \begin{macro}{\@endfloat@hook}
% \changes{v0.2}{2011/08/25}{Added macro}
%   This is done at the beginning of a float:
%    \begin{macrocode}
\def\@endfloat@hook{}
%    \end{macrocode}   
% \end{macro}
%
%
% \begin{macro}{\@nonfloat}
% \changes{v0.2}{2011/08/25}{Rewrote}
% \changes{v0.4}{2011/08/29}{Colors}
%   The main environment for setting up the the nonfloats.  The
%   argument is the type of the float
%    \begin{macrocode}
\def\@nonfloat#1{\@beginfloat@hook
  \ifKV@fao@fixedfloatheight
    \begin{minipage}[t][\@floatheight]{\@floatwidth}\begin{list}{}{%
          \topsep=\z@\partopsep=\z@\leftmargin=\z@\rightmargin=\z@}%
        \def\@captype{#1}\item
        \global\let\default@color\current@color\normalcolor
  \else
    \begin{minipage}{\@floatwidth}\begin{list}{}{%
          \topsep=\z@\partopsep=\z@\leftmargin=\z@\rightmargin=\z@}%
        \def\@captype{#1}\item
        \global\let\default@color\current@color\normalcolor
  \fi}
%    \end{macrocode}   
% \end{macro}
%
% \begin{macro}{\end@nonfloat}
% \changes{v0.2}{2011/08/25}{Rewrote}
% And ending the environment:
%    \begin{macrocode}
\def\end@nonfloat{\end{list}\end{minipage}%
\ifKV@fao@fixedfloatheight\else\vspace{\belowdisplayskip}\fi
\@endfloat@hook}
%    \end{macrocode}
% \end{macro}
%
% This environment makes it easy to redefine other environments:
% \begin{macro}{table}
%   Table:
%    \begin{macrocode}
\renewenvironment{table}{\begin{@nonfloat}{table}}{\end{@nonfloat}}
%    \end{macrocode}   
% \end{macro}
% \begin{macro}{chart}
%   Chart:
%    \begin{macrocode}
\renewenvironment{chart}{\begin{@nonfloat}{chart}}{\end{@nonfloat}}
%    \end{macrocode}   
% \end{macro}
% \begin{macro}{minitab}
%   Chart:
%    \begin{macrocode}
\renewenvironment{minitab}{\begin{@nonfloat}{minitab}}{\end{@nonfloat}}
%    \end{macrocode}   
% \end{macro}
% \begin{macro}{map}
%   Map:
%    \begin{macrocode}
\renewenvironment{map}{\begin{@nonfloat}{map}}{\end{@nonfloat}}
%    \end{macrocode}   
% \end{macro}
% 
%
% \changes{v0.11}{2011/09/22}{Added negative belowcaptionskip}
% \changes{v1.05}{2012/11/27}{Changed caption format}
% \changes{v1.06}{2012/12/05}{Added hyperlinks}
% Now we set up captions.  All our captions come \emph{before} the
% graphics material.  Note that due to the strange way
% \progname{caption} works this means that \cmd{\aboveskip} is actually
% the skip \emph{below} the caption.  Go figure.
%    \begin{macrocode}
\DeclareCaptionLabelFormat{uppercase}{\hyperlink{list:\@captype}{\MakeUppercase{#1}~#2}}
\captionsetup{position=top,labelformat=uppercase,format=plain,
  textfont=bf,justification=centering,
  singlelinecheck=false}
\captionsetup{aboveskip=\z@}
\AtEndCaption{\par}
%    \end{macrocode}
% 
%
% \begin{macro}{\ifKV@fao@fixedcaptionheight}
% \changes{v0.5}{2011/09/08}{Added macro}
%   Sometimes we typeset the caption in a box:
%    \begin{macrocode}
\define@boolkey{fao}{fixedcaptionheight}{%
  \ifKV@fao@fixedcaptionheight
    \def\caption@parbox{\parbox[t][\@fao@captionheight]}%
  \else
    \def\caption@parbox{\parbox[t]}%
  \fi}
\faoset{fixedfloatheight=false}
%    \end{macrocode}
% \end{macro}
%
% \begin{macro}{\@fao@captionheight}
% \changes{v0.5}{2011/09/08}{Added macro}
%   The height of the caption in fixed caption height mode:
%    \begin{macrocode}
\define@key{fao}{captionheight}{\def\@fao@captionheight{#1}}
\faoset{captionheight=2\baselineskip}
%    \end{macrocode}
% \end{macro}
%
% \changes{v0.8}{2011/09/13}{Deleted default graphicswidth}
% \begin{macro}{\footnotebar}
% \changes{v0.5}{2011/09/08}{Added macro}
% \changes{v0.6}{2011/09/09}{Added footnotesize decl}
%   A rule used mostly in footnotes:
%    \begin{macrocode}
\def\footnotebar{\par\vskip\skip\@mpfootins\footnoterule\footnotesize}
%    \end{macrocode}
%   
% \end{macro}
%
% \changes{v0.6}{2011/09/09}{Added H and P columntypes}
% \changes{v0.10}{2011/09/19}{Added d columntype}
% \changes{v0.20}{2011/10/06}{Changed color width in H, C, P column types}
% We define new column types for table headers:
%    \begin{macrocode}
\newcolumntype{d}[1]{D{.}{.}{#1}}
\newcolumntype{H}{>{\columncolor{@tableheadcolor}[1.01\tabcolsep][1.01\tabcolsep]}c}
%    \end{macrocode}
% 
%
% \changes{v0.10}{2011/09/20}{Added C columntype} 
% \changes{v0.10}{2011/09/20}{Rewrote P columntype}
% |P| columntype is
% much more complex.  Basically we want a centered 
% entry with a parbox
% of the given width inside.:
%    \begin{macrocode}
\newcolumntype{P}[1]{>{\columncolor{@tableheadcolor}[1.01\tabcolsep][1.01\tabcolsep]%
    \@fao@Pentry{#1}}c<{\end@fao@Pentry}}
%    \end{macrocode}
% \begin{macro}{\@fao@Pentry}
% \changes{v0.10}{2011/09/20}{Added macro}
% \changes{v0.10}{2011/10/03}{Added strut}
%   Since |\parbox| needs ``real'' braces to delimit the argument, 
%   we use this trick.  Note |\hspace{0pt}| to allow \TeX{} to
%   hyphenate the first word. 
%    \begin{macrocode}
\def\@fao@Pentry#1#2\end@fao@Pentry{%
\parbox[t]{#1}{\centering\strut\hspace{\z@}#2\strut}}
%    \end{macrocode}
%   
% \end{macro}
%
% Same with |C| entry:
%    \begin{macrocode}
\newcolumntype{C}[1]{>{\columncolor{@tableheadcolor}[1.01\tabcolsep][1.01\tabcolsep]%
    \@fao@Centry{#1}}c<{\end@fao@Centry}}
%    \end{macrocode}
% \begin{macro}{\@fao@Centry}
% \changes{v0.10}{2011/09/20}{Added macro}
% \changes{v0.10}{2011/10/03}{Added strut}
%   This macro is similar to |\@fao@Pentry|, but with different
%   way to set the width of the |\parbox|:
%    \begin{macrocode}
\def\@fao@Centry#1#2\end@fao@Centry{%
\settowidth{\@tempdima}{$-99.999$}%
\@tempdima=#1\@tempdima\relax
\parbox[t]{\@tempdima}{\centering\strut\hspace{\z@}#2\strut}}
%    \end{macrocode}
%   
% \end{macro}
%
% \begin{macro}{\tablepages}
% \changes{v0.8}{2011/09/13}{Added macro}
% \changes{v0.9}{2011/09/14}{Added coloring}
% \changes{v0.24}{2011/11/16}{Deleted coloring}
% \changes{v0.31}{2011/11/26}{Added page numbers}
% \changes{v0.42}{2012/11/14}{Added check for alt margins}
% \changes{v0.43}{2012/11/15}{Added check for alt margins narrow}
% \changes{v1.02}{2012/11/21}{Integrated changes for margins}
%   The special style for table pages:
%    \begin{macrocode}
\def\tablepages{\clearpage\bgroup\normalcolor
  \ifprint
    \if@altMargins
    \newgeometry{layout=a4paper,
      layouthoffset=5mm,layoutvoffset=5mm,
      left=2cm,right=1.5cm,bottom=1.7cm,top=.7cm,twoside}%
    \else\if@altMarginsNarrow
    \newgeometry{layout=a4paper,
      layouthoffset=5mm,layoutvoffset=5mm,
      left=1.5cm,right=1cm,bottom=1.7cm,top=.7cm,twoside}%
    \else
    \newgeometry{layout=a4paper,
      layouthoffset=1cm,layoutvoffset=1cm,
      left=0.2in,right=0.2in,top=0.2in,bottom=0.25in}%
    \fi\fi
  \else
  \newgeometry{layout=a4paper,
    left=0.2in,right=0.2in,top=0.2in,bottom=0.25in}%
  \fi
  \pagestyle{tablepage}\onecolumn}
%    \end{macrocode}
% \end{macro}
% \begin{macro}{\endtablepages}
% \changes{v0.8}{2011/09/13}{Added macro}
%   And reverting back:
%    \begin{macrocode}
\def\endtablepages{\egroup\loadgeometry{standard}%
  \pagestyle{standardpagestyle}\clearpage}
%    \end{macrocode}
%   
% \end{macro}
%
% \changes{v0.12}{2011/09/29}{Changed separation between the rules in
% tables}
% To prevent the ugly white spacese between the rules and the text we
% put the separations in \progname{booktabs} to zero.  To prevent the
% crammed tables we increase \cmd{\arraystretch}:
%    \begin{macrocode}
\aboverulesep\z@
\belowrulesep\z@
\setlength\minrowclearance{\z@}
\def\arraystretch{1.5}
%    \end{macrocode}
% 
% \begin{macro}{\arrayrulewidth}
% \changes{v0.15}{2011/10/12}{Changed the value}
% \changes{v0.20}{2011/11/06}{Increased}
%   This is the thickness of \cmd{\hhline}
%    \begin{macrocode}
\setlength{\arrayrulewidth}{0.9pt}
%    \end{macrocode}
% \end{macro}
%
% \begin{macro}{\vline}
% \changes{v0.15}{2011/10/12}{Changed the width}
% \changes{v0.20}{2011/11/06}{Increased width}
%   We use \cmd{\vline} for interrupting \cmd{\hhline} only.
%    \begin{macrocode}
\def\vline{\vrule \@width 7\p@ \relax}
%    \end{macrocode}
%   
% \end{macro}
%
%\subsection{Special Plot Layouts}
%\label{sec:special_plots}
%
% \begin{macro}{\TwoPlusFourPlots}
% \changes{v0.2}{2011/08/25}{Added macro}
% \changes{v0.5}{2011/09/05}{Added fixed caption height}
% \changes{v0.25}{2011/11/21}{Added \cmd{\pagebreak}}
%   This is simple: we make fixed plot heights:
%    \begin{macrocode}
\def\TwoPlusFourPlots{\bgroup
  \faoset{fixedfloatheight=true,floatheight=0.45\textheight}%
  \faoset{fixedcaptionheight=true,captionheight=2\baselineskip}}%
\def\endTwoPlusFourPlots{\pagebreak\egroup}
%    \end{macrocode}
% \end{macro}
%
%
% \begin{macro}{\@fao@plotnum}
% \changes{v0.2}{2011/08/25}{Added macro}
%   The counter to keep the current plot number
%    \begin{macrocode}
\newcount\@fao@plotnum
%    \end{macrocode}
% \end{macro}
%
% \begin{macro}{\TwoPlusTwoPlots}
% \changes{v0.2}{2011/08/25}{Added macro}
% \changes{v0.5}{2011/09/05}{Added fixed caption height}
% \changes{v0.5}{2011/09/05}{Added maximal figure height}
% \changes{v0.6}{2011/09/09}{Deleted maximal figure height}
%   This is more tricky than $2+4$: we need to keep the current plot
%   number and switch to one column when needed.
%    \begin{macrocode}
\def\TwoPlusTwoPlots{\bgroup
  \faoset{fixedfloatheight=true,floatheight=0.45\textheight}%
  \faoset{fixedcaptionheight=true,captionheight=2\baselineskip}%
  \global\@fao@plotnum=0\relax
  \def\@beginfloat@hook{\global\advance\@fao@plotnum by 1\relax
    \ifnum\@fao@plotnum=3\relax\onecolumn\fi}}
%    \end{macrocode}
% \end{macro}
% \begin{macro}{\endTwoPlusTwoPlots}
% \changes{v0.2}{2011/08/25}{Added macro}
%   \dots and switch back:
%    \begin{macrocode}
\def\endTwoPlusTwoPlots{\egroup\twocolumn}
%    \end{macrocode}
% \end{macro}
%
% \begin{macro}{\OnePlusTwoPlots}
% \changes{v0.9}{2011/09/14}{Added macro}
%   Here we want one tall float and two wide floats:
%    \begin{macrocode}
\def\OnePlusTwoPlots{\bgroup
  \faoset{fixedfloatheight=true,floatheight=0.95\textheight}%
  \faoset{fixedcaptionheight=true,captionheight=2\baselineskip}%
  \global\@fao@plotnum=0\relax
  \def\@beginfloat@hook{\global\advance\@fao@plotnum by 1\relax
    \ifnum\@fao@plotnum=2\relax\onecolumn\fi
    \ifnum\@fao@plotnum>1\faoset{floatheight=0.45\textheight}\fi}}
%    \end{macrocode}
%   
% \end{macro}
%
% \begin{macro}{\endOnePlusTwoPlots}
% \changes{v0.9}{2011/09/14}{Added macro}
%   \dots and switch back:
%    \begin{macrocode}
\def\endOnePlusTwoPlots{\egroup\twocolumn}
%    \end{macrocode}
% \end{macro}
%
%
%
%\subsection{Patching \progname{graphicx} Package}
%\label{sec:rviewport}
% \changes{v0.3}{2011/08/26}{Added rviewport code}
%
% Package \progname{graphicx} provides a useful keyword |viewport|
% which allows to show just a part of an image.  However, you need to
% put there the actual coordinates of the viewport window.  For large
% maps we would like to have relative coordinates as fractions of
% natural width size.  Here we define the new option |rviewport| for
% Relative Viewport.  It works like this.  Suppose the image has the
% bounding box $x_{ll}$, $y_{ll}$, $x_{ur}$, $y_{ur}$.  We give four
% numbers $\xi_{ll}$, $\eta_{ll}$, $\xi_{ur}$, $\eta_{ur}$, and the viewport
% coordinates become
% \begin{align*}
%   x'_{ll} &= x_{ll} + \xi_{ll}\Delta_x\\
%   y'_{ll} &= y_{ll} + \eta_{ll}\Delta_y\\
%   x'_{ur} &= x_{ll} + \xi_{ur}\Delta_x\\
%   y'_{ur} &= y_{ll} + \eta_{ur}\Delta_y
% \end{align*}
% where
% \begin{align*}
%   \Delta_x &= x_{ur} - x_{ll}\\
%   \Delta_y &= y_{ur} - y_{ll}
% \end{align*}
% This means that the left half of the image is |rviewport = 0 0 0.5 1|,
% and the right half is  |rviewport = 0.5 0 1 1|.
%
% We follow~\cite{Carlisle05:Graphics}.
%    \begin{macrocode}
\define@key{Gin}{rviewport}
           {\let\Gin@viewport@code\Gin@rviewport\Gread@parse@rvp#1 \\}
%    \end{macrocode}
% 
% \begin{macro}{\Gread@parse@rvp}
% \changes{v0.3}{2011/08/26}{Added macro}
%   We parse four numbers into the corresponding macros.  Note that
%   their names are significant:  |pdftex.def| would not clip the
%   image if |\Gin@vllx| is not defined.
%    \begin{macrocode}
\def\Gread@parse@rvp#1 #2 #3 #4 #5\\{%
  \def\Gin@vllx{#1}%
  \def\Gin@vlly{#2}%
  \def\Gin@vurx{#3}%
  \def\Gin@vury{#4}%
}%
%    \end{macrocode}
%   
% \end{macro}
%
% 
%
% \begin{macro}{\Gin@rviewport}
% \changes{v0.3}{2011/08/26}{Added macro}
%   And the viewport code.  Note that |pdftex.def| relies on
%   the values of |\Gin@v...| macros, so we redefine them as well. 
%    \begin{macrocode}
\def\Gin@rviewport{%
  \let\Gin@ollx\Gin@llx
  \let\Gin@olly\Gin@lly
  \let\Gin@ourx\Gin@urx
  \let\Gin@oury\Gin@ury
  \Gin@nat@width\Gin@urx\p@
  \advance\Gin@nat@width-\Gin@llx\p@
  \Gin@nat@height\Gin@ury\p@
  \advance\Gin@nat@height-\Gin@lly\p@
  \dimen@\Gin@vurx\Gin@nat@width
                      \edef\Gin@vurx{\strip@pt\dimen@}%
  \advance\dimen@\Gin@llx\p@
                      \edef\Gin@urx{\strip@pt\dimen@}%
  \dimen@\Gin@vury\Gin@nat@height
                      \edef\Gin@vury{\strip@pt\dimen@}%
  \advance\dimen@\Gin@lly\p@
                      \edef\Gin@ury{\strip@pt\dimen@}%
  \dimen@\Gin@vllx\Gin@nat@width
                      \edef\Gin@vllx{\strip@pt\dimen@}%
  \advance\dimen@\Gin@llx\p@
                    \edef\Gin@llx{\strip@pt\dimen@}%
  \dimen@\Gin@vlly\Gin@nat@height
                      \edef\Gin@vlly{\strip@pt\dimen@}%
  \advance\dimen@\Gin@lly\p@
                     \edef\Gin@lly{\strip@pt\dimen@}%
}
%    \end{macrocode}
%   
% \end{macro}
%
% \changes{v0.9}{2011/08/14}{Added global setting of keepaspectratio}
% We want all plots to keep the aspect ratio of the originals:
%    \begin{macrocode}
\setkeys{Gin}{keepaspectratio=true}
%    \end{macrocode}
% 
%
% \begin{macro}{\Gin@getbase}
% \changes{v0.15}{2011/10/12}{Redefined the macro}
%   This command selects the file for \cmd{\includegraphics}.  We
%   patch it to select print or web version if they exist:
%    \begin{macrocode}
\def\Gin@getbase#1{%
  \ifprint\def\filename@add{_print}\else\def\filename@add{_web}\fi
  \edef\Gin@tempa{%
    \def\noexpand\@tempa####1#1\space{%
      \def\noexpand\Gin@base{####1}}}%
  \IfFileExists{\filename@area\filename@base\filename@add#1}%
  {\Gin@tempa
    \expandafter\@tempa\@filef@und
    \edef\Gin@ext{#1}}{%
    \IfFileExists{\filename@area\filename@base#1}%
    {\Gin@tempa
      \expandafter\@tempa\@filef@und
      \edef\Gin@ext{#1}}{}}}%
%    \end{macrocode}
% \end{macro}
%
%\subsection{Multipage Maps}
%\label{sec:large_maps}
%
% \begin{macro}{\largeGraphicsNotes}
% \changes{v0.16}{2011/10/28}{Introduced the macro}
%   This puts its argument in the storage for the processing:
%    \begin{macrocode}
\long\def\largeGraphicsNotes#1{%
  \long\gdef\@largeGraphicsNotes{#1}}
\largeGraphicsNotes{}
%    \end{macrocode}
%   
% \end{macro}
%
% \begin{macro}{\@largegraphics@footnotes}
% \changes{v0.4}{2011/08/29}{Added macro}
%   Our |\@nonfloat| is a minipage.  Since |\includeLargeGraphics|
%   splits it, we need some place to store footnote captions.
%    \begin{macrocode}
\newinsert\@largegraphics@footnotes
%    \end{macrocode}   
% \end{macro}
%
% \begin{macro}{\Gin@leftfraction}
% \changes{v0.16}{2011/10/27}{Introduced the macro}
% \changes{v0.17}{2011/10/30}{Moved gdef to def}
%   This key might occur as the argument to
%   \cmd{\includeLargeGraphics} or \cmd{\includeExtraLargeGraphics}.
%   Since we want to be able to use there ``normal'' arguments, we put
%   it into the |Gin| family.
%    \begin{macrocode}
\define@key{Gin}{leftfraction}{\def\Gin@leftfraction{#1}}
\setkeys{Gin}{leftfraction=0.415}
%    \end{macrocode}
% \end{macro}
%
%
%
% \begin{macro}{\Gin@leftpartoffset}
% \changes{v0.17}{2011/10/30}{Introduced the macro}
%   The offset for the left part of the includegraphics
%    \begin{macrocode}
\newdimen\Gin@leftpartoffset
\define@key{Gin}{leftpartoffset}{\Gin@leftpartoffset=#1\relax}
\setkeys{Gin}{leftpartoffset=-87pt}
%    \end{macrocode}   
% \end{macro}
%
% \begin{macro}{\Gin@rightpartoffset}
% \changes{v0.17}{2011/10/30}{Introduced the macro}
%   The offset for the right part of the includegraphics
%    \begin{macrocode}
\newdimen\Gin@rightpartoffset
\define@key{Gin}{rightpartoffset}{\Gin@rightpartoffset=#1\relax}
\setkeys{Gin}{rightpartoffset=0pt}
%    \end{macrocode}   
% \end{macro}
%
%
% \begin{macro}{\if@extra@large@graphics}
% \changes{v0.16}{2011/10/27}{Introduced the macro}
%   This is false inside large graphics, but true inside large
%   graphics.
%    \begin{macrocode}
\newif\if@extra@large@graphics
\@extra@large@graphicsfalse
%    \end{macrocode}
%   
% \end{macro}
%
% \begin{macro}{\@includeLargeGraphics}
% \changes{v0.16}{2011/10/27}{Introduced the macro}
%   This is the generic macro for  \cmd{\includeLargeGraphics} or
%   \cmd{\includeExtraLargeGraphics}:
%    \begin{macrocode}
\def\@includeLargeGraphics{\@ifnextchar[% ] To make Emacs happy
   {\@@includeLargeGraphics}{\@@includeLargeGraphics[]}}
%    \end{macrocode}
% \end{macro}
%
% \begin{macro}{\@@includeLargeGraphics}
% \changes{v0.16}{2011/10/27}{Introduced the macro}
% \changes{v0.16}{2011/10/28}{Added notes}
% \changes{v0.16}{2011/10/28}{Added check for empty footnotes}
% \changes{v0.16}{2011/10/28}{Added afterpage before switching to
% twocolumn}
% \changes{v0.17}{2011/10/30}{Moved setkeys to inside the minipages}
% \changes{v0.17}{2011/10/30}{Added offsets}
%   This is the work horse of the (extra) large graphcis\dots
%    \begin{macrocode}
\def\@@includeLargeGraphics[#1]#2{%
%    \end{macrocode}
%   First we read the keys:
%    \begin{macrocode}
  \setkeys{Gin}{#1}%
%    \end{macrocode}
% We need to check whether we are in a float environment and in the
% proper column mode:
%    \begin{macrocode}
  \ifx\@captype\@undefined\PackageError{faoyearbook}{%
    Wrong place for \string\includeLargeGraphics{} or
    \string\includeExtraLargeGraphics}{%
    The commands \string\includeLargeGraphics{} and 
    \string\includeExtraLargeGraphics{} should be inside
    map or chart environment}\fi
  \if@extra@large@graphics
    \if@twocolumn\PackageError{faoyearbook}{%
    Wrong place for \string\includeExtraLargeGraphics}{%
    The command \string\includeExtraLargeGraphics{} should be inside
    a one column page}\fi
  \else
    \if@twocolumn\else\PackageError{faoyearbook}{%
    Wrong place for \string\includeLargeGraphics}{%
    The command \string\includeLargeGraphics{} should be inside
    a two column page}\fi
  \fi
%    \end{macrocode}
% We save the footnotes so far
%    \begin{macrocode}
 \ifvoid\@mpfootins\else
   \global\setbox\@largegraphics@footnotes\vbox{\unvbox\@mpfootins}%
 \fi
%    \end{macrocode}
%
%  Now we close the minipage and remember the vertical offset.  We
%  also want to restore two column mode after we close \emph{if} we
%  are not in ``extra large graphics'':
%    \begin{macrocode}
  \end{list}\end{minipage}%
  \if@extra@large@graphics
    \def\end@nonfloat{\end{list}\end{minipage}}%
  \else
    \def\end@nonfloat{\end{list}\end{minipage}\afterpage{\twocolumn}}%
  \fi
  \par
  \@tempdima=-\topskip
  \advance\@tempdima\pagetotal
%    \end{macrocode}
% 
%  We are ready to print the left part of the image and the notes.
%  Note that we re-read the keys inside the minipage
%    \begin{macrocode}
    \ifvoid\@largegraphics@footnotes\else
      \global\setbox\@mpfootins\vbox{%
        \unvbox\@largegraphics@footnotes}%
    \fi
    \begin{minipage}{\@floatwidth}\begin{list}{}{%
          \setkeys{Gin}{#1}%
          \topsep=\z@
          \partopsep=\z@\leftmargin=\z@\rightmargin=\z@}%
          \let\default@color\current@color\normalcolor\item   
          \hspace*{\Gin@leftpartoffset}%
          \includegraphics[rviewport=0 0 {\Gin@leftfraction} 1,clip]{#2}\par
          \@largeGraphicsNotes
%    \end{macrocode}
%  Reset the notes:
%    \begin{macrocode}
         \largeGraphicsNotes{}%
   \end{list}\end{minipage}%
%    \end{macrocode}
%  
%   Now go to the next page and put the right part:
%    \begin{macrocode}
    \if@extra@large@graphics\newpage\else\onecolumn\fi
    \vspace*{\@tempdima}%
    \begin{minipage}{\@floatwidth}\begin{list}{}{%
          \setkeys{Gin}{#1}%
          \topsep=\z@
          \partopsep=\z@\leftmargin=\z@\rightmargin=\z@}%
          \let\default@color\current@color\normalcolor\item
          \hspace*{\Gin@rightpartoffset}%
          \includegraphics[rviewport={\Gin@leftfraction} 0 1 1,clip]{#2}}
%    \end{macrocode}
%
% \end{macro}
%
% \begin{macro}{\includeLargeGraphics}
% \changes{v0.3}{2011/08/26}{Added macro}
% \changes{v0.4}{2011/08/29}{Changed the treatment of footnotes}
% \changes{v0.16}{2011/10/27}{Moved to @includeLargeGraphics}
% \changes{v0.18}{2011/11/04}{Moved offsets}
%   Include three column graphics:
%    \begin{macrocode}
\def\includeLargeGraphics{%
  \@extra@large@graphicsfalse
  \setkeys{Gin}{leftfraction=0.415, leftpartoffset=-87\p@}%
  \@includeLargeGraphics}
%    \end{macrocode}
% \end{macro}
%
% \begin{macro}{\includeExtraLargeGraphics}
% \changes{v0.3}{2011/08/26}{Added macro}
% \changes{v0.4}{2011/08/29}{Changed the treatment of footnotes}
% \changes{v0.16}{2011/10/27}{Moved to @includeLargeGraphics}
%   Include four column graphics
%    \begin{macrocode}
\def\includeExtraLargeGraphics{%
  \@extra@large@graphicstrue
  \setkeys{Gin}{leftfraction=0.5}%
  \@includeLargeGraphics}
%    \end{macrocode}
% \end{macro}
%
% \begin{macro}{\graphicKey}
% \changes{v0.16}{2011/10/28}{Moved to @includeLargeGraphics}
%   This typesets everything in a box in the lower right corner
%    \begin{macrocode}
\newenvironment{graphicKey}[1][]{\vfill\par\hfill
  \def\@tempa{#1}\ifx\@empty\@tempa\else\begin{minipage}{#1}\fi}{%
    \ifx\@tempa\@empty\else\end{minipage}\fi}
%    \end{macrocode}
%   
% \end{macro}
%
%\subsection{Formatting of Table of Contents and Lists}
%\label{sec:toc}
%
% \begin{macro}{\tableofcontents}
% \changes{v0.7}{2011/09/11}{Redefined macro}
% \changes{v0.7}{2011/09/11}{Moved leftskip}
% \changes{v0.29}{2011/11/25}{Added \cmd{\clerdoublepage}}
%   Table of contents is formatted in a special way:
%    \begin{macrocode}
\renewcommand\tableofcontents{\cleardoublepage
  \makebox[0pt][l]{\fontsize{24pt}{32pt}\selectfont \bfseries
    \color{black!70}\MakeUppercase{\contentsname}\space}%
  \par\vspace{-2\baselineskip}\vspace{-\parskip}%
  \@starttoc{toc}}
%    \end{macrocode}
% \end{macro}
%
% 
% \begin{macro}{\@tocpartskip}
% \changes{v0.33}{2011/12/06}{Introduced the macro}
%   This is the skip between the parts in TOC:
%    \begin{macrocode}
\newlength{\@tocpartskip}
\define@key{fao}{tocpartskip}{\setlength{\@tocpartskip}{#1}}
\faoset{tocpartskip=\z@}
%    \end{macrocode}
%   
% \end{macro}
%
% \begin{macro}{\@fao@tocrule@start}
% \changes{v0.7}{2011/09/11}{Added macro}
%   The start of the current TOC colored rule
%    \begin{macrocode}
\newdimen\@fao@tocrule@start
%    \end{macrocode}
%   
% \end{macro}
%
% 
% \begin{macro}{\@fao@tocrule@height}
% \changes{v0.7}{2011/09/11}{Added macro}
%   The height of the current TOC rule
%    \begin{macrocode}
\newdimen\@fao@tocrule@height
%    \end{macrocode}
%   
% \end{macro}
%
% \begin{macro}{\@draw@tocrule@part}
% \changes{v0.7}{2011/09/11}{Added macro}
% \changes{v0.9}{2011/09/14}{Rewrote}
% \changes{v0.12}{2011/09/28}{Added negative \cmd{\vspace}}
% \changes{v0.22}{2011/11/14}{Increased height by 3pt}
% \changes{v0.24}{2011/11/16}{Decreased width}
% \changes{v0.36}{2011/12/26}{Added writing position to the aux file}
% \changes{v1.01}{2012/11/21}{Integrated changes of margins}
%   Drawing the toc rule for a part
%    \begin{macrocode}
\def\@draw@tocrule@part{\@fao@tocrule@height=\pagetotal
  \protected@write\@auxout{}{\string\fao@partblobbottom{\fao@currentpartnum}{\the\@fao@tocrule@height}}%  
  \advance\@fao@tocrule@height-\@fao@tocrule@start
  \bgroup\parskip\z@
  \parbox[b][\z@]{\z@}{\hspace*{-15\p@}\color{@bgcolor}\rule{2\p@}{\@fao@tocrule@height}}%
  \parbox[b][\z@]{\z@}{\hspace*{330\p@}%
    \if@altMarginsNarrow\hspace{10mm}\fi
    \color{@bgcolor}\rule{41\p@}{\@fao@tocrule@height}}%
  \par\vspace{-0.5\baselineskip}\egroup}
%    \end{macrocode}
%   
% \end{macro}
%
%
% \begin{macro}{\@draw@tocrule@section}
% \changes{v0.7}{2011/09/11}{Added macro}
% \changes{v0.9}{2011/09/14}{Rewrote}
% \changes{v0.11}{2011/09/22}{Changed dimensions}
% \changes{v0.24}{2011/11/16}{Decreased width}
% \changes{v0.27}{2011/11/23}{Added zero dimensions}
% \changes{v0.36}{2011/12/26}{Added writing position to the aux file}
% \changes{v1.01}{2012/11/21}{Integrated changes of margins}
%   Drawing the toc rule for a section
%    \begin{macrocode}
\def\@draw@tocrule@section{\@fao@tocrule@height=\pagetotal
  \protected@write\@auxout{}{\string\fao@partblobbottom{\fao@currentpartnum}{\the\@fao@tocrule@height}}%  
  \advance\@fao@tocrule@height-\@fao@tocrule@start
  \advance\@fao@tocrule@height5\p@
  \bgroup\parskip\z@\small
  \raisebox{\baselineskip}[\z@][\z@]{\parbox[b][\z@]{\z@}{\hspace*{-35\p@}\color{@bgcolor}\rule{2\p@}{\@fao@tocrule@height}}}%
  \raisebox{\baselineskip}[\z@][\z@]{\parbox[b][\z@]{\z@}{\hspace*{310\p@}%
    \if@altMarginsNarrow\hspace{10mm}\fi
      \color{@bgcolor}\rule{41\p@}{\@fao@tocrule@height}}}%
  \par\vspace{-\baselineskip}\egroup}
%    \end{macrocode}
%   
% \end{macro}
%
% \begin{macro}{\l@part}
% \changes{v0.7}{2011/09/11}{Redefined macro}
% \changes{v0.8}{2011/09/13}{Moved leftskip}
% \changes{v0.9}{2011/09/14}{Moved drawing rule after par}
% \changes{v0.33}{2011/12/06}{Added tocpartskip}
% \changes{v0.36}{2011/12/19}{Added writing start to the aux}
% \changes{v0.44}{2012/11/18}{Added check for AltMarginsNarrow}
%   This prints the part in TOC:
%    \begin{macrocode}
\renewcommand*\l@part[2]{%
  \ifnum \c@tocdepth >-2\relax
    \addpenalty{-\@highpenalty}%
    \setlength\@tempdima{3em}%
    \addvspace{\@tocpartskip}%
    \begingroup
%    \end{macrocode}
% We store the current vertical position of the page into
% \cmd{\@fao@tocrule@start} 
%    \begin{macrocode}
%     \addvspace{-2pc}\par
     \@fao@tocrule@start=\pagetotal
     \protected@write\@auxout{}{\string\fao@partblobtop{\fao@currentpartnum}{\the\@fao@tocrule@start}}%
      \parindent \z@ \rightskip \@pnumwidth
      \parfillskip -\@pnumwidth
      \leftskip180\p@
      {\leavevmode
       \color{@bgcolor}\bfseries\partname\space#1:
       \hfil \hb@xt@\@pnumwidth{\hss #2}}%
     \par\@draw@tocrule@part
       \nobreak
         \global\@nobreaktrue
         \everypar{\global\@nobreakfalse\everypar{}}%
    \endgroup
  \fi}
%    \end{macrocode}
% \end{macro}
%
% \begin{macro}{\l@spart}
%   This adds unnumbered part to TOC
% \changes{v0.26}{2011/11/22}{Added macro}
% \changes{v0.36}{2011/12/19}{Added writing start to the aux}
% \changes{v0.44}{2012/11/18}{Added check for AltMarginsNarrow}
%    \begin{macrocode}
\newcommand*\l@spart[2]{%
  \ifnum \c@tocdepth >-2\relax
    \addpenalty{-\@highpenalty}%
    \setlength\@tempdima{3em}%
    \begingroup
     \@fao@tocrule@start=\pagetotal
     \protected@write\@auxout{}{\string\fao@partblobtop{\fao@currentpartnum}{\the\@fao@tocrule@start}}%
      \parindent \z@ \rightskip \@pnumwidth
      \parfillskip -\@pnumwidth
      \leftskip180\p@
      {\leavevmode
       \color{@bgcolor}\bfseries#1:
       \hfil \hb@xt@\@pnumwidth{\hss #2}}%
     \par\@draw@tocrule@part
       \nobreak
         \global\@nobreaktrue
         \everypar{\global\@nobreakfalse\everypar{}}%
    \endgroup
  \fi}
%    \end{macrocode}
% 
% \end{macro}
%
%
% \begin{macro}{\l@section}
% \changes{v0.7}{2011/09/11}{Redefined macro}
% \changes{v0.8}{2011/09/13}{Moved leftskip}
% \changes{v0.9}{2011/09/14}{Moved drawing rule after par}
% \changes{v0.11}{2011/09/22}{Added parskip zero}
% \changes{v0.27}{2011/11/23}{Added strut}
% \changes{v0.36}{2011/12/19}{Added writing to aux file}
% \changes{v0.44}{2012/11/18}{Added check for AltMarginsNarrow}
%   This prints the section in TOC:
%    \begin{macrocode}
\renewcommand*\l@section[2]{%
  \ifnum \c@tocdepth >-2\relax
    \addpenalty{-\@highpenalty}%
    \setlength\@tempdima{3em}%
    \begingroup 
    \small
     \@fao@tocrule@start=\pagetotal
     \leftskip200\p@\relax\parskip\z@
      \parindent \z@ \rightskip \@pnumwidth
      \parfillskip -\@pnumwidth
      {\leavevmode\small\strut
       #1\hfil \hb@xt@\@pnumwidth{\hss #2}}\par\@draw@tocrule@section
       \nobreak
         \global\@nobreaktrue
         \everypar{\global\@nobreakfalse\everypar{}}%
    \endgroup
  \fi}
%    \end{macrocode}
%   
% \end{macro}
%
% \begin{macro}{\appendix}
% \changes{v0.14}{2011/10/10}{Redefined macro}
% \changes{v0.38}{2012/02/05}{Bookmarks start at root}
%   We do not draw colored rules in the TOC part of the appendix:
%    \begin{macrocode}
\renewcommand\appendix{%
  \bookmarksetup{startatroot}%
  \addtocontents{toc}{\string\let\string\@draw@tocrule@part\string\relax
    \string\let\string\@draw@tocrule@section\string\relax}}
%    \end{macrocode}
%   
% \end{macro}
%
% 
% \changes{v1.10}{2013/02/26}{Added writing par to the lists}
% We use special formatting of metadata in the lists of\dots{} This
% requires explicit |\pars| at the end:
%    \begin{macrocode}
\AtEndDocument{%
  \immediate\write\@auxout{\string\@writefile{loc}{\string\par}}%
  \immediate\write\@auxout{\string\@writefile{lot}{\string\par}}%
  \immediate\write\@auxout{\string\@writefile{lom}{\string\par}}}
%    \end{macrocode}
% 
%
%\subsection{Metadata}
%\label{sec:metadata}
%
% \begin{macro}{\metadata}
% \changes{v0.14}{2011/10/10}{Added macro}
% \changes{v0.31}{2011/11/26}{Changed penalty}
% \changes{v0.37}{2012/01/14}{Changed penalties}
% \changes{v0.39}{2012/02/19}{Added support for nameref}
% \changes{v1.07}{2012/12/09}{Rewrote}
% \changes{v1.09}{2012/12/18}{Decreased topsep}
%   This starts the metadata section.  The commands inside are local
%   to the metadata.
%    \begin{macrocode}
\def\metadata#1#2{\bgroup
  \def\meta@key{#2}%
%    \end{macrocode}
% \begin{macro}{\source}
% \changes{v0.14}{2011/10/10}{Added macro}
% \changes{v0.37}{2012/01/14}{Changed penalties}
%   This typesets the source:
%    \begin{macrocode}
    \def\source##1{\par\penalty10000\emph{Source: }##1\par\penalty10000}%
%    \end{macrocode}   
% \end{macro}
% \begin{macro}{\owner}
% \changes{v0.14}{2011/10/10}{Added macro}
% \changes{v0.37}{2012/01/14}{Changed penalties}
%   This typesets the owner:
%    \begin{macrocode}
    \def\owner##1{\par\penalty10000\emph{Owner: }##1\par\penalty10000}%
%    \end{macrocode}   
% \end{macro}
%
%
%     \begin{macrocode}
  \begin{list}{}{\topsep4\p@\labelwidth\z@
      \labelsep\z@\itemindent\z@\parsep0.4ex plus 0.5ex minus
      0.2ex\relax\listparindent\z@\leftmargin\z@\rightmargin\z@
    \partopsep\z@}%
  \NR@gettitle{#1}\phantomsection\label{metadata:#2}%
  \item{\bfseries#1\par\penalty10000}}
%    \end{macrocode}
% \end{macro}
% \begin{macro}{\endmetadata}
% \changes{v0.14}{2011/10/10}{Added macro}
% \changes{v1.07}{2012/12/09}{Rewrote}
%   This closes the environment:
%    \begin{macrocode}
\def\endmetadata{%
  \expandafter\ifx\csname
     metaback@\meta@key\endcsname\relax
  \else
    \par\penalty10000\emph{Referenced in: }
    \csname metaback@\meta@key\endcsname
     \par\penalty10000
  \fi
  \end{list}\egroup}
%    \end{macrocode}
%   
% \end{macro}
%
%
% \begin{macro}{\refMetadataname}
% \changes{v1.07}{2012/12/09}{Added macro}
%   The name for the metadata reference:
%    \begin{macrocode}
\def\refMetadataname{Metadata:}
%    \end{macrocode}
%   
% \end{macro}
%
% \begin{macro}{\refMetadata}
% \changes{v1.07}{2012/12/09}{Added macro}
% \changes{v1.08}{2012/12/15}{Made invisible}
% \changes{v1.08}{2012/12/15}{Added support for longtables}
%   The way we actually reference the metadata:
%    \begin{macrocode}
\def\refMetadata#1{%
  \ifx\@captype\@undefined\def\@captype{table}\fi
    \if@filesw
      \immediate\write\@mainaux{%
        \string\fao@metaback{#1}{\@captype}{\csname the\@captype\endcsname}{\thepage}{\@currentHref}}%
    \addtocontents{\csname ext@\@captype\endcsname}{\string\listmetadata{#1}}%
    \fi
%  \refMetadataname~\nameref{metadata:#1}, page~\pageref{metadata:#1}
}
%    \end{macrocode}
%   
% \end{macro}
%
% \begin{macro}{\fao@metaback}
% \changes{v1.07}{2012/12/09}{Added macro}
%   This reads the backreferences to metadata and prepares the the
%   list.  The arguments are: key, float type, number of float, page
%   and hyperref
%    \begin{macrocode}
\def\fao@metaback#1#2#3#4#5{%
  \expandafter\ifx\csname metaback@#1\endcsname\relax
    \expandafter\gdef\csname metaback@#1\endcsname{%
    \hyper@linkstart{link}{#5}#2~#3\hyper@linkend}%
  \else
    \expandafter\g@addto@macro\csname metaback@#1\endcsname{, 
    \hyper@linkstart{link}{#5}#2~#3\hyper@linkend}%
  \fi}
%    \end{macrocode}
%   
% \end{macro}
%
% \begin{macro}{\ifKV@fao@metadataInLists}
% \changes{v1.07}{2012/12/09}{Added macro}
%   Whether put metadata in lists of\dots
%    \begin{macrocode}
\define@boolkey{fao}{metadataInLists}{}
\faoset{metadataInLists=false}
%    \end{macrocode}   
% \end{macro}
%
% \begin{macro}{\listmetadata}
% \changes{v1.07}{2012/12/09}{Added macro}
% \changes{v1.08}{2012/12/15}{Added check for multiple sources}
% \changes{v1.09}{2012/12/18}{Made smaller font and the word
% ``Metadata'' in italics}
% \changes{v1.10}{2013/01/26}{Changed formatting}
%   The way metadata is presented in the lists:
%    \begin{macrocode}
\def\listmetadata#1{\ifKV@fao@metadataInLists
  \bgroup\small
  \ifvmode\relax
    \leavevmode
    \textit{\refMetadataname}~%
   \else
     \unskip;~%
  \fi
  \nameref{metadata:#1}, page~\pageref{metadata:#1}%
  \egroup\hangafter=0\hangindent=3.8em\rightskip=3.8em\relax
  \fi}
%    \end{macrocode}
%   
% \end{macro}
%
% \begin{macro}{\MetadataCollection}
% \changes{v1.08}{2012/12/15}{Added macro}
%   The section with metadata
\newenvironment{MetadataCollection}{\clearpage\twocolumn\small
  \loadgeometry{standard}%
  \pagestyle{conceptpagestyle}}{\clearpage\pagestyle{standardpagestyle}\normalsize}
% \end{macro}
%
%\subsection{Concepts and Methods}
%\label{sec:concepts}
% 
% \begin{macro}{\conceptsname}
% \changes{v1.05}{2012/11/27}{Added macro}
%   The name for concepts section
%    \begin{macrocode}
\def\conceptsname{Concepts and Methods}
%    \end{macrocode}
%   
% \end{macro}
%
% \begin{macro}{\ConceptsAndMethods}
% \changes{v1.05}{2012/11/27}{Added macro}
%   This is the style for the concepts and methods
%    \begin{macrocode}
\newenvironment{ConceptsAndMethods}{\clearpage\twocolumn\small
  \loadgeometry{standard}%
  \pagestyle{conceptpagestyle}%
  \section{\conceptsname}}{\clearpage\pagestyle{standardpagestyle}\normalsize}
%    \end{macrocode}
%   
% \end{macro}
%
% \begin{macro}{\concept}
% \changes{v1.04}{2012/11/26}{Added macro}
% \changes{v1.05}{2012/11/27}{Rewrote}
%   This is the environment for concepts:
%    \begin{macrocode}
\newenvironment{concept}[1]{\begin{description}\item[#1:]}{\end{description}}
%    \end{macrocode}
%   
% \end{macro}
%
%
%\subsection{Further Reading}
%\label{sec:freading}
%
% \begin{macro}{\fitemize}
% \changes{v0.16}{2011/10/28}{Added macro}
%   This is the special version of |itemize| for further reading
%   pages.  Basically it is a patched kernel version.
%    \begin{macrocode}
\def\fitemize{%
  \ifnum \@itemdepth >\thr@@\@toodeep\else
    \advance\@itemdepth\@ne
    \edef\@itemitem{labelitem\romannumeral\the\@itemdepth}%
    \expandafter
    \list
      \csname\@itemitem\endcsname
      {\def\makelabel##1{\color{@bgcolor}{##1}\space}%
        \itemsep\z@\labelwidth\z@
        \leftmargin\z@\labelsep\z@}%
  \fi}
%    \end{macrocode}
% \end{macro}
% \begin{macro}{\endfitemize}
% \changes{v0.16}{2011/10/28}{Added macro}
%   This is standard:
%    \begin{macrocode}
\let\endfitemize =\endlist
%    \end{macrocode}
%   
% \end{macro}
% 
%
% \begin{macro}{\freading}
% \changes{v0.16}{2011/10/28}{Added macro}
% \changes{v0.19}{2011/11/04}{Changed vertical spacing}
% \changes{v0.19}{2011/11/04}{Changed subsection to section}
%   This is the ``Further Reading environment''
%    \begin{macrocode}
\newenvironment{freading}{%
  \vfill\section*{Further reading}\par
  \vspace{-\baselineskip}{\color{@bgcolor}{\rule{\columnwidth}{1.5pt}}}\par
  \vspace{-\baselineskip}\bgroup
  \let\itemize=\fitemize
  \let\enditemize=\endfitemize}{\egroup}
%    \end{macrocode}
%   
% \end{macro}
% 
%
%\subsection{Publication}
%\label{sec:publications}
%
% \begin{macro}{\@publicationskip}
% \changes{v0.34}{2011/12/06}{Introduced the macro}
%   Skip between the publications.  By default \cmd{\medskip}:
%    \begin{macrocode}
\newlength{\@publicationskip}
\define@key{fao}{publicationskip}{\setlength{\@publicationskip}{#1}}
\faoset{publicationskip=6pt plus 2pt minus 2pt}
%    \end{macrocode}
% \end{macro}
%
% \begin{macro}{\@publicationparskip}
% \changes{v0.34}{2011/12/06}{Introduced the macro}
%   Paragraph skip between the publications.
%    \begin{macrocode}
\newlength{\@publicationparskip}
\define@key{fao}{publicationparskip}{\setlength{\@publicationparskip}{#1}}
\faoset{publicationparskip=6pt plus 6pt minus 4pt}
%    \end{macrocode}
%   
% \end{macro}
%
% \begin{macro}{\publication}
% \changes{v0.21}{2011/11/09}{Introduced the environment}
% \changes{v0.22}{2011/11/14}{Added medskip}
% \changes{v0.34}{2011/12/06}{Added adjustable skips}
%   This typesets one publication:
%    \begin{macrocode}
\newenvironment{publication}[2][]{%
  \par{\bfseries#2\par}\begin{minipage}[t]{0.49\columnwidth}%
    \setlength\parskip{\@publicationparskip}%
    \gdef\@pub@cover{#1}%
    \long\def\pDescription##1{\par##1\par}%
    \def\pEdition##1##2{\par##1: ##2\par}%
    \def\pCycle##1{\par Publication cycle: ##1\par}%
    \def\pWeb##1{\par \raggedright Webpage: \url{##1}\par}}%
  {\end{minipage}%
  \ifx\@pub@cover\@empty\else
  \hspace{0.1\columnwidth}%
  \raisebox{\dimexpr\baselineskip-\totalheight}{%
    \includegraphics[width=0.4\columnwidth]{\@pub@cover}}\fi\par
  \vspace{\@publicationskip}}
%    \end{macrocode}
%   
% \end{macro}
%
%
%\subsection{Subscripts}
%\label{sec:subscripts}
%
% \begin{macro}{\textsubscript}
%   This follows standard \LaTeX:
% \changes{v0.21}{2011/11/11}{Introduced the environment}
%    \begin{macrocode}
\DeclareRobustCommand*\textsubscript[1]{%
  \@textsubscript{\selectfont#1}}
\def\@textsubscript#1{%
  {\m@th\ensuremath{_{\mbox{\fontsize\sf@size\z@#1}}}}}
%    \end{macrocode}
% 
% \end{macro}
%
%
%
%\subsection{LyX code}
%\label{sec:lyx}
% 
% \begin{macro}{\lyxlist}
% \changes{v0.30}{2011/11/25}{Added list code required by Lyx}
% It seems Lyx wants this:
%    \begin{macrocode}
\newenvironment{lyxlist}[1]
{\begin{list}{}
{\settowidth{\labelwidth}{#1}
 \setlength{\leftmargin}{\labelwidth}
 \addtolength{\leftmargin}{\labelsep}
 \renewcommand{\makelabel}[1]{##1\hfil}}}
{\end{list}}
%    \end{macrocode}
%   
% \end{macro}
%
%
%
% \subsection{The final word}
%\label{sec:final}
%
%    \begin{macrocode}
\twocolumn
%</class>      
%    \end{macrocode}
%   
%\Finale
%\clearpage
%
%\PrintChanges
%\clearpage
%\PrintIndex
%
\endinput
